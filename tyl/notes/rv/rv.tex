\documentclass{article}
\usepackage{fullpage}
\title{A Syntax and Semantics for\\Probability and Statistics with Random Variables}
\author{Sooraj Bhat}

%\setlength{\jot}{3.0\jot}
\parskip 0.7pc
\parindent 0pc

\newcommand{\E}[1]{\ensuremath{#1}}
\newcommand{\T}[1]{\texttt{#1}}
\newcommand{\rbool}{\T{rbool}}
\newcommand{\rint}{\T{rint}}
\newcommand{\rreal}{\T{rreal}}
\newcommand{\true}{\T{true}}
\newcommand{\false}{\T{false}}
\newcommand{\rlet}[4]{\E{\T{rlet}\ #1 : #2 \sim #3 \ \T{in}\ #4}}
\newcommand{\pdf}[2]{\E{\T{pdf(} #1 \ | \ #2 \T{)}}}
\newcommand{\prob}[2]{\E{\T{prob(} #1 \ | \ #2 \T{)}}}
\newcommand{\thereis}[3]{\E{\exists\ #1 : #2 \ . \ #3}}
\newcommand{\Not}{\T{not}}
\newcommand{\argmax}[2]{\E{\T{argmax}_{#1}\ #2}}
\newcommand{\case}{\T{case}}
\newcommand{\ite}[3]{\E{\T{if}\ #1\ \T{then}\ #2\ \T{else}\ #3}}
\newcommand{\tildepmf}{\E{\stackrel{\scriptscriptstyle\mathrm{pmf}}{\sim}}}
\newcommand{\judge}[2]{\E{\displaystyle\frac{#1}{#2}}}
\newcommand{\isoftype}[3]{#1\vdash #2 : #3}

\begin{document}
\maketitle
\section{Introduction}
%
why interesting, why unsolved, contributions. \tildepmf
%
\section{Syntax}
% 
\begin{eqnarray*}
t & ::= & \rbool \ | \ \rint \ | \ \rreal \\
%\end{eqnarray*}
& & \\
%\begin{eqnarray*}
e & ::= & x \ | \ r \ | \ \true \ | \ \false \\
& & \ | \ -e \ | \ e_1 + e_2 \ | \ e_1 - e_2 
    \ | \ e_1 * e_2 \ | \ e_1 / e_2 \ | \ e_1^{e_2} \ | \ \T{log}\ e \\
& & \ | \ \Not\ e \ | \ e_1 \vee e_2 \ | \ e_1 \wedge e_2 \\
& & \ | \ e_1 \leq e_2 \ | \ e_1 = e_2 \ | \ e_1 \geq e_2 \\
& & \ | \ \ite{e_1}{e_2}{e_3} \\
& & \ | \ \rlet{x}{t}{e_1}{e_2} 
\end{eqnarray*}

Revelation: variables don't have values, context-senstive evaluation semantics.

Thought experiement, (X+Y)+(X+Y) vs 2*(X+Y)

\rint, and \rreal\ are synonyms for $[-\infty,\infty]$ and
$<-\infty,\infty>$.  For now, we are disallowing arbitrary {\tt let}s
in the object language to solve the RV 'identity' problem.  Also, we
disallow random variable literals (they would only be marginally
useful) ... RVs can only be introduced with \T{rlet}.

must make distinction based on discrete vs.\ continuous, not int vs real, e.g.
\rlet{x}{\rint}{\T{UniformInt(1,6)}}{\sqrt{x}} ... the distribution is still
discrete even though the support are not integers.

Notes: multivariate/joint distributions can (for now) be represented
by the equivalent cascade of conditionals (i.e.\ $p(x,y,z) =
p(x)p(y|x)p(z|x,y)$), hidden parameters (a.k.a.\ existentially
quantified variables: \thereis{x}{t}{e}) can often be obviated (see
2-Gaussian example, below),

Things to be added: \prob{e_1}{e_2}, argmax, \pdf{e_1}{e_2},
indexing. \T{prob} is a special case of \T{pdf}.

Currently, no distinction is made between random reals and normal
reals, so that things like $f(x) = x^2 + 5$ operate on both.

\section{Examples}
%
\[ \rlet{x,y,z}{\rreal}{\mathrm{Normal}(0,1),\mathrm{Poisson}(x),\mathrm{Bernoulli}(3/4)}
       {\prob{x+3\leq y}{x=y \ \vee \ \ite{z=0}{\true}{\false}}} \]
%
\[ \rlet{x,y,z}{\rreal}{\_,\_,\pdf{x}{y}}{\_} \]
%
\[ \thereis{\mu,\sigma}{\rreal}{\rlet{x}{\rreal}{N(\mu,\sigma)}{\prob{x\leq x^2}{\_}}} \]
%
\[ \argmax{\mu_0,\mu_1,\sigma_0,\sigma_1:\rreal}{ %
  \rlet{c_i,x_i}{\rint,\rreal}{\mathrm{Bernoulli}(1/2),\mathrm{Normal}(
    \ite{c_i=0}{\mu_0}{\mu_1},\ite{c_i=0}{\sigma_0}{\sigma_1})}
       {\pdf{\bigwedge_i x_i=r_i }{\_}}
} \]

\begin{verbatim}
-- j in [0..1], i in [1..n]
exists mu[j], sigma[j] : real in
rlet c[i] ~ Bernoulli(1/2) in
rlet x[i] ~ Normal(mu[c[i]],sigma[c[i]]) in
argmax_{mu',sigma'} pdf( {x[i]=r[i]} | mu=mu' /\ sigma=sigma' )
\end{verbatim}

\section{Type System}

\rint\ is a refinement on \rreal.

type checking for the arithmetic operations are on the coarse type, \rreal.  

\judge{(x:\tau)\in\Gamma}{\isoftype{\Gamma}{x}{\tau}}
\judge{}{\isoftype{\Gamma}{\true}{\rbool}}
\judge{}{\isoftype{\Gamma}{\false}{\rbool}}
\judge{\isoftype{\Gamma}{e}{\rreal}}{\isoftype{\Gamma}{-e}{\rreal}}

\section{Semantics}

\section{Things to support}

\begin{itemize}
\item probability
\item expectation
\item variance 
\item joint $\rightarrow$ marginalize
\item joint 
\item inequalities
\item goodness-of-fit, normality tests
\item order statistics
\item bounds , hoeffding's bound , specific dist. $\rightarrow$ special bounds
\item confidence intervals, hypothesis testing
\item likelihood, posterior, prior, evidence
\item convergence
\item sampling
\item mean, median, mode
\end{itemize}

\section{Related Work}

\begin{itemize}
\item PFP. Erwig.
\item APPL. Diane Evans.
\item Probability monads.
\item CES. Thrun.
\item Tyl. Agarwal.
\end{itemize}


\section{References}
\end{document}
