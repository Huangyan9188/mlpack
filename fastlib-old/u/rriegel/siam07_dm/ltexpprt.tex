% This is ltexpprt.tex, an example file for use with the SIAM LaTeX2E
% Preprint Series macros. It is designed to provide double-column output. 
% Please take the time to read the following comments, as they document
% how to use these macros. This file can be composed and printed out for
% use as sample output.

% Any comments or questions regarding these macros should be directed to:
%
%                 Donna Witzleben
%                 SIAM
%                 3600 University City Science Center
%                 Philadelphia, PA 19104-2688
%                 USA
%                 Telephone: (215) 382-9800
%                 Fax: (215) 386-7999
%                 e-mail: witzleben@siam.org


% This file is to be used as an example for style only. It should not be read
% for content.

%%%%%%%%%%%%%%% PLEASE NOTE THE FOLLOWING STYLE RESTRICTIONS %%%%%%%%%%%%%%%

%%  1. There are no new tags.  Existing LaTeX tags have been formatted to match
%%     the Preprint series style.    
%%
%%  2. You must use \cite in the text to mark your reference citations and 
%%     \bibitem in the listing of references at the end of your chapter. See
%%     the examples in the following file. If you are using BibTeX, please
%%     supply the bst file with the manuscript file.
%% 
%%  3. This macro is set up for two levels of headings (\section and 
%%     \subsection). The macro will automatically number the headings for you.
%%
%%  5. No running heads are to be used for this volume.
%% 
%%  6. Theorems, Lemmas, Definitions, etc. are to be double numbered, 
%%     indicating the section and the occurence of that element
%%     within that section. (For example, the first theorem in the second
%%     section would be numbered 2.1. The macro will 
%%     automatically do the numbering for you.
%%
%%  7. Figures, equations, and tables must be single-numbered. 
%%     Use existing LaTeX tags for these elements.
%%     Numbering will be done automatically.
%%   
%%
%%%%%%%%%%%%%%%%%%%%%%%%%%%%%%%%%%%%%%%%%%%%%%%%%%%%%%%%%%%%%%%%%%%%%%%%%%%%%%%



\documentclass[twoside,leqno,twocolumn]{article}  
\usepackage{ltexpprt} 

\begin{document}


%\setcounter{chapter}{2} % If you are doing your chapter as chapter one,
%\setcounter{section}{3} % comment these two lines out.

\title{\Large SIAM/ACM Preprint Series Macros for
Use With LaTeX\thanks{Supported by GSF grants ABC123, DEF456, and GHI789.}}
\author{Corey Gray\thanks{Society for Industrial and Applied Mathematics.} \\
\and 
Tricia Manning\thanks{Society for Industrial and Applied Mathematics.}}
\date{}

\maketitle

 
%\pagenumbering{arabic}
%\setcounter{page}{1}%Leave this line commented out.

\begin{abstract} \small\baselineskip=9pt This is the text of my abstract. It is a brief
description of my
paper, outlining the purposes and goals I am trying to address.\end{abstract}

 


\section{Problem Specification.}In this paper, we consider the solution of the $N \times
N$ linear
system
\begin{equation} \label{e1.1}
A x = b
\end{equation}
where $A$ is large, sparse, symmetric, and positive definite.  We consider
the direct solution of (\ref{e1.1}) by means of general sparse Gaussian
elimination.  In such a procedure, we find a permutation matrix $P$, and
compute the decomposition
\[
P A P^{t} = L D L^{t}
\]
where $L$ is unit lower triangular and $D$ is diagonal.  

 
\section{Design Considerations.}Several good ordering algorithms (nested dissection and
minimum degree)
are available for computing $P$  \cite{GEORGELIU}, \cite{ROSE72}.
Since our interest here does not
focus directly on the ordering, we assume for convenience that $P=I$,
or that $A$ has been preordered to reflect an appropriate choice of $P$.

Our purpose here is to examine the nonnumerical complexity of the
sparse elimination algorithm given in  \cite{BANKSMITH}.
As was shown there, a general sparse elimination scheme based on the
bordering algorithm requires less storage for pointers and
row/column indices than more traditional implementations of general
sparse elimination.  This is accomplished by exploiting the m-tree,
a particular spanning tree for the graph of the filled-in matrix.

\begin{theorem} The method  was extended to three
dimensions. For the standard multigrid
coarsening
(in which, for a given grid, the next coarser grid has $1/8$
as many points), anisotropic problems require plane
relaxation to
obtain a good smoothing factor.\end{theorem} 

Our purpose here is to examine the nonnumerical complexity of the
sparse elimination algorithm given in  \cite{BANKSMITH}.
As was shown there, a general sparse elimination scheme based on the
bordering algorithm requires less storage for pointers and
row/column indices than more traditional implementations of general
sparse elimination.  This is accomplished by exploiting the m-tree,
a particular spanning tree for the graph of the filled-in matrix.
Several good ordering algorithms (nested dissection and minimum degree)
are available for computing $P$  \cite{GEORGELIU}, \cite{ROSE72}.
Since our interest here does not
focus directly on the ordering, we assume for convenience that $P=I$,
or that $A$ has been preordered to reflect an appropriate choice of $P$.

\begin{proof} In this paper we consider two methods. The first method
is
basically the method considered with two differences:
first, we perform plane relaxation by a two-dimensional
multigrid method, and second, we use a slightly different
choice of
interpolation operator, which improves performance
for nearly singular problems. In the second method coarsening
is done by successively coarsening in each of the three
independent variables and then ignoring the intermediate
grids; this artifice simplifies coding considerably.
\end{proof}

Our purpose here is to examine the nonnumerical complexity of the
sparse elimination algorithm given in  \cite{BANKSMITH}.
As was shown there, a general sparse elimination scheme based on the
bordering algorithm requires less storage for pointers and
row/column indices than more traditional implementations of general
sparse elimination.  This is accomplished by exploiting the m-tree,
a particular spanning tree for the graph of the filled-in matrix.

\begin{Definition}{\rm We describe the two methods in \S 1.2. In \S\ 1.3. we
discuss
some remaining details.}
\end{Definition}

Our purpose here is to examine the nonnumerical complexity of the
sparse elimination algorithm given in  \cite{BANKSMITH}.
As was shown there, a general sparse elimination scheme based on the
bordering algorithm requires less storage for pointers and
row/column indices than more traditional implementations of general
sparse elimination.  This is accomplished by exploiting the m-tree,
a particular spanning tree for the graph of the filled-in matrix.
Several good ordering algorithms (nested dissection and minimum degree)
are available for computing $P$  \cite{GEORGELIU}, \cite{ROSE72}.
Since our interest here does not
focus directly on the ordering, we assume for convenience that $P=I$,
or that $A$ has been preordered to reflect an appropriate choice of $P$.

Our purpose here is to examine the nonnumerical complexity of the
sparse elimination algorithm given in  \cite{BANKSMITH}.
As was shown there, a general sparse elimination scheme based on the
bordering algorithm requires less storage for pointers and
row/column indices than more traditional implementations of general
sparse elimination.  

\begin{lemma} We discuss first the choice for $I_{k-1}^k$
which is a generalization. We assume that $G^{k-1}$ is
obtained
from $G^k$
by standard coarsening; that is, if $G^k$ is a tensor product
grid $G_{x}^k \times G_{y}^k \times G_{z}^k$,
$G^{k-1}=G_{x}^{k-1} \times G_{y}^{k-1} \times G_{z}^{k-1}$,
where $G_{x}^{k-1}$ is obtained by deleting every other grid
point of $G_x^k$ and similarly for $G_{y}^k$ and $G_{z}^k$.
\end{lemma}
 
To our knowledge, the m-tree previously has not been applied in this
fashion to the numerical factorization, but it has been used,
directly or indirectly, in several optimal order algorithms for
computing the fill-in during the symbolic factorization phase
[4] - [10], [5], [6]. In \S 1.3., we analyze the complexity of the old and new
approaches to the intersection problem for the special case of
an $n \times n$ grid ordered by nested dissection. The special
structure of this problem allows us to make exact estimates of
the complexity. To our knowledge, the m-tree previously has not been applied in this
fashion to the numerical factorization, but it has been used,
directly or indirectly, in several optimal order algorithms for
computing the fill-in during the symbolic factorization phase
[4] - [10], [5], [6].

In \S 1.2, we review the bordering algorithm, and introduce
the sorting and intersection problems that arise in the
sparse formulation of the algorithm.  
In \S 1.3., we analyze the complexity of the old and new
approaches to the intersection problem for the special case of
an $n \times n$ grid ordered by nested dissection. The special
structure of this problem allows us to make exact estimates of
the complexity. To our knowledge, the m-tree previously has not been applied in this
fashion to the numerical factorization, but it has been used,
directly or indirectly, in several optimal order algorithms for
computing the fill-in during the symbolic factorization phase
[4] - [10], [5], [6].


For the old approach, we show that the
complexity of the intersection problem is $O(n^{3})$, the same
as the complexity of the numerical computations.  For the
new approach, the complexity of the second part is reduced to
$O(n^{2} (\log n)^{2})$.  

To our knowledge, the m-tree previously has not been applied in this
fashion to the numerical factorization, but it has been used,
directly or indirectly, in several optimal order algorithms for
computing the fill-in during the symbolic factorization phase
[4] - [10], [5], [6]. In \S 1.3., we analyze the complexity of the old and new
approaches to the intersection problem for the special case of
an $n \times n$ grid ordered by nested dissection. The special
structure of this problem allows us to make exact estimates of
the complexity. To our knowledge, the m-tree previously has not been applied in this
fashion to the numerical factorization, but it has been used,
directly or indirectly, in several optimal order algorithms for
computing the fill-in during the symbolic factorization phase
[4] - [10], [5], [6].
This is accomplished by exploiting the m-tree,
a particular spanning tree for the graph of the filled-in matrix.
To our knowledge, the m-tree previously has not been applied in this
fashion to the numerical factorization, but it has been used,
directly or indirectly, in several optimal order algorithms for
computing the fill-in during the symbolic factorization phase
\cite{EISENSTAT} - \cite{LIU2}, \cite{ROSE76},  \cite{SCHREIBER}.

\subsection{Robustness.}We do not
attempt to present an overview
here, but rather attempt to focus on those results that
are relevant to our particular algorithm.
This section assumes prior knowledge of the role of graph theory
in sparse Gaussian elimination; surveys of this role are
available in \cite{ROSE72} and \cite{GEORGELIU}. More general
discussions of elimination trees are given in
\cite{LAW} - \cite{LIU2}, \cite{SCHREIBER}.
Thus, at the $k$th stage, the bordering algorithm consists of
solving the lower triangular system
\begin{equation} \label{1.2}
 L_{k-1}v = c
\end{equation}
and setting
\begin{eqnarray} 
\ell &=& D^{-1}_{k-1}v , \\
\delta &=& \alpha - \ell^{t} v .
\end{eqnarray}

\begin{figure}
\vspace{14pc}
\caption{This is a figure 1.1.}
\end{figure}

\section{Robustness.} We do not
attempt to present an overview
here, but rather attempt to focus on those results that
are relevant to our particular algorithm.
 
\subsection{Versatility.}The special
structure of this problem allows us to make exact estimates of
the complexity.  For the old approach, we show that the
complexity of the intersection problem is $O(n^{3})$, the same
as the complexity of the numerical computations
\cite{GEORGELIU}, \cite{ROSEWHITTEN}.  For the
new approach, the complexity of the second part is reduced to
$O(n^{2} (\log n)^{2})$. 

To our knowledge, the m-tree previously has not been applied in this
fashion to the numerical factorization, but it has been used,
directly or indirectly, in several optimal order algorithms for
computing the fill-in during the symbolic factorization phase
[4] - [10], [5], [6]. In \S 1.3., we analyze the complexity of the old and new
approaches to the intersection problem for the special case of
an $n \times n$ grid ordered by nested dissection. The special
structure of this problem allows us to make exact estimates of
the complexity. To our knowledge, the m-tree previously has not been applied in this
fashion to the numerical factorization, but it has been used,
directly or indirectly, in several optimal order algorithms for
computing the fill-in during the symbolic factorization phase
[4] - [10], [5], [6].

In \S 1.2, we review the bordering algorithm, and introduce
the sorting and intersection problems that arise in the
sparse formulation of the algorithm.  
In \S 1.3., we analyze the complexity of the old and new
approaches to the intersection problem for the special case of
an $n \times n$ grid ordered by nested dissection. The special
structure of this problem allows us to make exact estimates of
the complexity. To our knowledge, the m-tree previously has not been applied in this
fashion to the numerical factorization, but it has been used,
directly or indirectly, in several optimal order algorithms for
computing the fill-in during the symbolic factorization phase
[4] - [10], [5], [6].


For the old approach, we show that the
complexity of the intersection problem is $O(n^{3})$, the same
as the complexity of the numerical computations.  For the
new approach, the complexity of the second part is reduced to
$O(n^{2} (\log n)^{2})$.  

To our knowledge, the m-tree previously has not been applied in this
fashion to the numerical factorization, but it has been used,
directly or indirectly, in several optimal order algorithms for
computing the fill-in during the symbolic factorization phase
[4] - [10], [5], [6]. In \S 1.3., we analyze the complexity of the old and new
approaches to the intersection problem for the special case of
an $n \times n$ grid ordered by nested dissection. The special
structure of this problem allows us to make exact estimates of
the complexity. To our knowledge, the m-tree previously has not been applied in this
fashion to the numerical factorization, but it has been used,
directly or indirectly, in several optimal order algorithms for
computing the fill-in during the symbolic factorization phase
[4] - [10], [5], [6].
This is accomplished by exploiting the m-tree,
a particular spanning tree for the graph of the filled-in matrix.
To our knowledge, the m-tree previously has not been applied in this
fashion to the numerical factorization, but it has been used,
directly or indirectly, in several optimal order algorithms for
computing the fill-in during the symbolic factorization phase
\cite{EISENSTAT} - \cite{LIU2}, \cite{ROSE76},  \cite{SCHREIBER}.

\begin{thebibliography}{99}

%\bibitem{GUIDE}
%R.~E. Bank, {\em PLTMG  users' guide, edition 5.0}, tech. report,
%  Department of Mathematics, University of California, San Diego, CA, 1988.

%\bibitem{HBMG}
%R.~E. Bank, T.~F. Dupont, and H.~Yserentant, {\em The hierarchical basis
%  multigrid method}, Numer. Math., 52 (1988), pp.~427--458.

\bibitem{BANKSMITH}
R.~E. Bank and R.~K. Smith, {\em General sparse elimination requires no
  permanent integer storage}, SIAM J. Sci. Stat. Comput., 8 (1987),
  pp.~574--584.

\bibitem{EISENSTAT}
S.~C. Eisenstat, M.~C. Gursky, M.~Schultz, and A.~Sherman, {\em
  Algorithms and data structures for sparse symmetric gaussian elimination},
  SIAM J. Sci. Stat. Comput., 2 (1982), pp.~225--237.

\bibitem{GEORGELIU}
A.~George and J.~Liu, {\em Computer Solution of Large Sparse Positive
  Definite Systems}, Prentice Hall, Englewood Cliffs, NJ, 1981.

\bibitem{LAW}
K.~H. Law and S.~J. Fenves, {\em A node addition model for symbolic
  factorization}, ACM TOMS, 12 (1986), pp.~37--50.

\bibitem{LIU}
J.~W.~H. Liu, {\em A compact row storage scheme for cholesky factors
  using elimination trees}, ACM TOMS, 12 (1986), pp.~127--148.

\bibitem{LIU2}
\sameauthor , {\em The role of
  elimination trees in sparse factorization}, Tech. Report CS-87-12,Department
  of Computer Science, York University, Ontario, Canada, 1987.

\bibitem{ROSE72}
D.~J. Rose, {\em A graph theoretic study of the numeric solution of
  sparse positive definite systems}, in Graph Theory and Computing, Academic  Press, New
York, 1972.

\bibitem{ROSE76}
D.~J. Rose, R.~E. Tarjan, and G.~S. Lueker, {\em Algorithmic aspects of
  vertex elimination on graphs}, SIAM J. Comput., 5 (1976), pp.~226--283.

\bibitem{ROSEWHITTEN}
D.~J. Rose and G.~F. Whitten, {\em A recursive analysis of disection
  strategies}, in Sparse Matrix Computations, Academic Press, New York, 1976.

\bibitem{SCHREIBER}
R.~Schrieber, {\em A new implementation of sparse gaussian elimination},
  ACM TOMS, 8 (1982), pp.~256--276.

\end{thebibliography}
\end{document}
