\documentclass{article}
\usepackage{nips07submit_e,times}
%\documentstyle[nips07submit_09,times]{article}

\usepackage{amsmath,amsthm,amssymb}


\newtheorem{theorem}{Theorem}
\newtheorem{lemma}{Lemma}
\newtheorem{corollary} {Corollary}
\newtheorem{definition} {Definition}
\newtheorem{example} {Example}


\newcommand{\GNP}[1][\psi]{{#1}_{\Theta}}
\newcommand{\GNPvec}[1][\psi]{{#1}_{\overrightarrow{\Theta}}}

\newcommand{\otimesvec}{\mathbin{\overrightarrow{\otimes}}}
\newcommand{\odotvec}{\mathbin{\overrightarrow{\odot}}}
\newcommand{\bigotimesvec}{\mathop{\overrightarrow{\bigotimes}}}
\newcommand{\bigodotvec}{\mathop{\overrightarrow{\bigodot}}}

\newcommand{\otimeshat}{\mathbin{\widehat{\otimes}}}
\newcommand{\odothat}{\mathbin{\widehat{\odot}}}
\newcommand{\bigotimeshat}{\mathop{\widehat{\bigotimes}}}
\newcommand{\bigodothat}{\mathop{\widehat{\bigodot}}}

\newcommand{\otimestilde}{\mathbin{\widetilde{\otimes}}}
\newcommand{\odottilde}{\mathbin{\widetilde{\odot}}}
\newcommand{\bigotimestilde}{\mathop{\widetilde{\bigotimes}}}
\newcommand{\bigodottilde}{\mathop{\widetilde{\bigodot}}}

\DeclareMathOperator*{\argmin}{argmin}
\DeclareMathOperator*{\argmax}{argmax}
\DeclareMathOperator*{\map}{map}
\DeclareMathOperator*{\minmax}{minmax}
\DeclareMathOperator{\sign}{sign}

\DeclareMathOperator{\leftchild}{left}
\DeclareMathOperator{\rightchild}{right}
\DeclareMathOperator{\parent}{parent}
\DeclareMathOperator{\sibling}{sibling}
\DeclareMathOperator{\op}{op}

\newcommand{\comp}{\mathbin{\circ}}
\newcommand{\st}{{\rm~s.t.~}}

% Useful commands
\newcommand{\fig}[1]{Figure~\ref{fig:#1}}

% Standardized pseudocode functions
%\newcommand{\dist}[2]{||#1-#2||}
\newcommand{\spos}{^{{\scriptscriptstyle +\!}}}
\newcommand{\sneg}{^{{\scriptscriptstyle -\!}}}

\newcommand{\disthrectmin}{d^{l}}
\newcommand{\disthrectmax}{d^{u}}
\newcommand{\dist}[2]{d(#1,#2)}
\newcommand{\kdroot}[1]{#1^{\!\text{root}}}
\newcommand{\kdleft}[1]{#1^{\!L}}
\newcommand{\kdright}[1]{#1^{\!R}}

\newcommand{\al}{a^l}
\newcommand{\au}{a^u}
\newcommand{\dl}{d^l}
\newcommand{\du}{d^u}

% Spacing for standardized pseudocode
\newcommand{\x}{\\ \hspace{0.15in}}
\newcommand{\xx}{\\ \hspace{0.3in}}
\newcommand{\xxx}{\\ \hspace{0.45in}}
\newcommand{\xxxx}{\\ \hspace{0.6in}}

% Variables for affinity propagation
\newcommand{\eqspace}{\!\!\!\!}
\newcommand{\cpos}[2]{c^{+}_{#1 \neq #2}}
\newcommand{\cneg}[2]{c^{-}_{#1 \neq #2}}

\newcommand{\intersect}{\cap}

\newcommand{\respo}[2]{R_{#1#2}}
\newcommand{\avail}[2]{A_{#1#2}}
\newcommand{\simil}[2]{S_{#1#2}}

\newcommand{\vecrho}{\vec{\rho}}
\newcommand{\vecalpha}{\vec{\alpha}}
\newcommand{\frho}[1]{\rho_{#1}}
\newcommand{\falpha}[1]{\alpha_{#1}}
\newcommand{\falphaj}[2]{\alpha_{#1[#2]}}

\newcommand{\falphamax}{\alpha^{u}}
\newcommand{\falphamin}{\alpha^{l}}
\newcommand{\frhomax}{\rho^{u}}
\newcommand{\frhomin}{\rho^{l}}

\newcommand{\alphacand}{v}

\newcommand{\ignoreall}[1]{}

\title{Some Awesome Title}

\author{
Ryan N.~Riegel \\
College of Computiong \\
Georgia Institute of Technology \\
Atlanta, GA 30332 \\
\texttt{rriegel@cc.gatech.edu} \\
\And
Alexander G.~Gray \\
College of Computiong \\
Georgia Institute of Technology \\
Atlanta, GA 30332 \\
\texttt{agray@cc.gatech.edu} \\
}

% The \author macro works with any number of authors. There are two commands
% used to separate the names and addresses of multiple authors: \And and \AND.
%
% Using \And between authors leaves it to \LaTeX{} to determine where to break
% the lines. Using \AND forces a linebreak at that point. So, if \LaTeX{}
% puts 3 of 4 authors names on the first line, and the last on the second
% line, try using \AND instead of \And before the third author name.

\begin{document}

\makeanontitle

\begin{abstract}
We present mathematical foundations for a highly successful
algorithmic strategy that has resulted in the fastest algorithms for
many machine learning methods and is broadly applicable to scaling
many future methods up to large datasets.  We formalize for the first
time a class of computational problems which are very common in
machine learning, called {\em generalized $N$-body problems}, and
subsequently present a template {\em generalized $N$-body algorithm},
which can be specialized to produce efficient problem-dependent
algorithms.  We demonstrate the use of this mathematical framework for
deriving the recent affinity propagation method.
\end{abstract}

\section{Introduction}

The ability to apply machine learning methods to large datasets is
increasingly important for many applications.  Unfortunately,
generally speaking, the more accurate the method, the greater its
computational cost; most nonparametric methods are $O(N^2)$ or
$O(N^3)$, where there are $O(N)$ test (or query) points and training
(or reference) points.

The authors of [[nips2000paper]] presented a new algorithmic strategy
based on the simultaneous traversal of multiple space-partitioning
trees, which applies to a number of machine learning methods.  The set
of problems to which this strategy is applicable was informally named
{\em generalized $N$-body problems}, in analogy to breakthrough
methods for computational physics problems [[Appel's algorithm,
Barnes-Hut, Fast Multipole Method]], all of which have the same
characteristic form.  This algorithmic strategy generalizes other
successful algorithms for specific problems in addition to these
physics problems, such as computing the well-separated pair
decomposition in theoretical computer science [[wspd]], computing the
spatial join in databases [[spatial join]], and computing set
intersections [[baeza-yates]].  In each of these problems no superior
algorithmic strategy is known.
% also a kind of dual-tree method in graphics for object collision
% detection, but we need the reference

This multi-tree algorithmic strategy has been applied to a succession
of well-known statistical learning methods, each representing certain
unique challenges for the strategy, including
all-$k$-nearest-neighbors (a generalization of $k$-nearest-neighbors)
[[nips2000paper]], kernel density estimation [[nips2000paper,
kde-siamdm, kde-aistats, kde-nips-dong, kde-uai-dong]],
$k$-nearest-neighbor classification [[ting-liu]], kernel discriminant
analysis (or nonparametric Bayes classification) [[nbc-phystat,
nbc-compstat]], and $n$-point correlation functions [[nips2000paper,
moore-npt, npt-2004]].  These algorithms have been demonstrated on
large scientific datasets, producing numerous high-profile results,
e.g.~[[science 2003, nature 2005]].  For each of these problems, no
overall faster algorithms are known.  In addition, other authors have
directed the strategy toward other machine learning problems,
including dimensionality reduction methods [[hochreiter00beyond]],
nonparametric belief propagation [[alex ihler]], multiple tracking
[[jeremy kubica]], linear algebraic machine learning methods [[nando,
below]], particle filters [[klaas papers, below]].  Aside from speed,
a unique advantage of this algorithmic approach is the ability to
specify and achieve rigorous relative error tolerances when
approximation is necessary, contrary to virtually all other
approximate speedup approaches.
%
%Nando de Freitas, Yang Wang, Maryam Mahdaviani and Dustin Lang. Fast
%Krylov Methods for N-Body Learning . NIPS 2005.
%
%Mike Klaas, Dustin Lang and Nando de Freitas. Fast Maximum a
%Posteriori Inference in Monte Carlo State Spaces . AISTATS 2005.
%
%Mike Klaas, Nando de Freitas and Arnaud Doucet. Toward Practical N^2
%Monte Carlo: The Marginal Particle Filter . UAI 2005
%
%Mike Klaas, Mark Briers, Nando de Freitas, Arnaud Doucet, Simon
%Maskell and Dustin Lang. Fast Particle Smoothing: If I Had a Million
%Particles. ICML 2006. 
%
%inproceedings{ hochreiter00beyond,
%    author = "Sepp Hochreiter and Michael Mozer",
%    title = "Beyond Maximum Likelihood and Density Estimation: A Sample-Based Criterion for Unsupervised Learning of Complex Models",
%    booktitle = "{NIPS}",
%    pages = "535-541",
%    year = "2000",
%    url = "citeseer.ist.psu.edu/541475.html" }

\section{Examples}

% Begin figure

\begin{figure}
  \begin{displaymath}
    \begin{array}{ll}
      \hspace{-0.17in}
      \begin{array}{l}
        \begin{array}{l}
          \text{function tpc}(Q, R)
          \x \text{if }\disthrectmax(X_1, X_2) < r\text{: return }0
          \x \text{if }\disthrectmin(X_1, X_1) > r\text{: return } |X_1| \cdot |X_2|
          \x \text{elif }|X_1| \geq |X_2|\text{:}
          \xx \text{return tpc}(\kdleft{X_1}, X_2, r) + \text{tpc}(\kdright{X_1}, X_2, r)
          \x \text{else:}
          \xx \text{return tpc}(X_1, \kdleft{X_2}, r) + \text{tpc}(X_1, \kdright{X_2}, r)
          %\\ \\ \\ \\
        \end{array}
        \\
        \\
        \begin{array}{l}
          \text{init all nodes }Q \subseteq \kdroot{Q}\text{: }a(Q) \gets \infty
          \\ \text{procedure allnn}(Q,R)\text{:}
          \x \text{if }a(Q) < \disthrectmin(Q, R)\text{: return}
          \x \text{elif }Q = \{q\} \text{ and } R = \{r\}
          \xx a(\{q\}) \gets \min(a(q), \dist{Q}{R})
          \x \text{elif }|Q| \geq |R|\text{:}
          \xx \text{allnn}(\kdleft{Q}, R); \text{ allnn}(\kdright{Q}, R)
          \xx a(Q) \gets \max(a(\kdleft{Q}), a(\kdright{Q}))
          \x \text{else prioritize by min distance:}
          \xx \text{allnn}(Q, \kdleft{R}); \text{ allnn}(Q, \kdright{R})
        \end{array}
      \end{array}
      & \hspace{-0.2in}
      \begin{array}{l}
        \\ \text{init all nodes }Q \subseteq \kdroot{Q}\text{: }a(Q) \gets (0, 0)
        \\ P \gets \text{new priority queue};~\text{postpone}(\kdroot{Q}, \kdroot{R})
        \\ \text{while } P \text{ not empty}
        \x (Q, R, \dl, \du) \gets \text{pop maximum from }P
        \x \forall Q' \subseteq Q,~ \au(Q') \gets \max(\au(\kdleft{Q}), \au(\kdright{Q}))
        \x \forall Q' \subseteq Q,~ \al(Q') \gets \min(\al(\kdleft{Q}), \al(\kdright{Q}))
        \x \text{if }\al(Q) > 0\text{: label all points in }Q\text{ positive}
        \x \text{elif }\au(Q) < 0\text{: label all points in }Q\text{ negative}
        \x \text{else:}
        \xx \text{recursively, }a(Q') \gets a(Q') - (\dl, \du)
        \xx \text{postpone}(\kdleft{Q}, \kdleft{R}); \text{ postpone}(\kdleft{Q}, \kdright{R})
        \xx \text{postpone}(\kdright{Q}, \kdleft{R}); \text{ postpone}(\kdright{Q}, \kdright{R})
        \\ \text{procedure postpone}(Q, R)\text{:}
        \x \dl \gets \pi\spos|R\spos|K^{u}(Q, R\spos) + \pi\sneg|R\sneg|K^{l}(Q, R\sneg)
        \x \du \gets \pi\spos|R\spos|K^{l}(Q, R\spos) + \pi\sneg|R\sneg|K^{u}(Q, R\sneg)
        \x \forall Q' \subset Q,~ a(Q') \gets a(Q') + (\dl, \du)
        \x \text{place }(Q, R, \dl, \du)\text{ in }P\text{ at priority }(\du-\dl)
      \end{array}
    \end{array}
  \end{displaymath}
  \label{fig:allnntpc}
  \caption{Pseudocode for three simple dual-tree algorithms: two-point correlation (tpc), all-nearest-neighbors (allnn), and nonparametric Bayes classification.}
\end{figure}

Here, we present three canonical dual-tree algorithms: two-point correlation, all-nearest-neighbors, and nonparametric Bayes classification.
We use spatial trees, where every node $X$ in tree $\kdroot{X}$ is a set of points; every internal node is partitioned $\kdleft{X} \cup \kdright{X} = X$, with each leaf a single point $\{x\}$.
We also denote $d$ to be a distance metric, and superscripts $l$ and $u$ to refer respectively to upper and lower bounds.

{\bf Two-point Correlation.} As a member of the family of $n$-point correlation functions, collectively the foundation for all spatial statistics, the two point correlation\footnote{For simplicity, we do not exclude redundant pairs; this is trivially handled algebraicly.} of data set $X$ for radius $r$ is:
\begin{equation*}
\sum_{x_1 \in X} \sum_{x_2 \in X} I(d(x_1, x_2) \leq r)
\end{equation*}
\noindent for indicator function $I$.
\fig{allnntpc} shows a formulation that recursively considers a subset pair $X_1$ and $X_2$, returning immediately if all points in $X_1$ and $X_2$ are completely inside or outside the radius.

{\bf All-nearest-neighbors.} In applications from manifold learning to computational physics, it is often desirable to find for a batch of queries $Q$ the nearest neighbors from reference set $R$:
\begin{equation*}
\sum_{q \in Q} \argmin_{r \in R} d(q,r)
\end{equation*}
\noindent with the exception $d(q,q) = \infty$.
To achieve speedup, one maintains for each recursive subset of queries the furthest candidate neighbor found.
If a considered set of references is farther away than this distance, no further exploration is required.

{\bf Nonparametric Bayes Classification.}
A simple kernel-based classifier labels each point in $Q$ positive or negative by weighing the density generated by a kernel at each training example in $R\spos$ and $R\sneg$, weighed respectively by priors $\pi\spos$ and $\pi\sneg$:
\begin{equation*}
\map_{q \in Q} I\Big(\sum_{r \in R\spos} \pi\spos K(q,r) - \sum_{r \in R\sneg} \pi\sneg K(q,r) \geq 0 \Big)
\end{equation*}
\noindent for distance-based kernel $K$.
Each subset of queries maintains an upper and lower bound on the difference of the sums, and is labelled when this range no longer contains zero.
A simple non-recursive algorithm uses a priority queue, relying on the fact that iterating over all children is amortized linear in quite inexpensive compared to the recursion\footnote{In practice, some of these updates may be performed lazily.}.

%  The all-nearest-neighbors problem (All-NN) aims to find for each
%  point in some set $Q$ of queries the nearest point from a set $R$ of
%  references, possibly identical to $Q$.  The problem is represented
%  mathematically as
%  \[
%  \map_{1 \leq i \leq |Q|}\argmin_{1 \leq j \leq |R|} d(q_i,r_j),
%  \]
%  where metric $d(q_i,r_i) \equiv \infty$ if $Q = R$.
%  
%  [[Examples of uses.]]
%  
%  All-NN is solved efficiently by Algorithm~\ref{alg:all-nn}, which
%  works on trees formed for both queries and references.  It maintains
%  for each query node an upper bound on the distance to any of its
%  points' nearest neighbors; in short, the maximum distance to any
%  candidate nearest neighbor seen so far.  When a pair of query and
%  reference nodes have a lower-bound distance greater than the query
%  node's stored upper bound, it is impossible for the reference node to
%  contribute any of the contained queries' nearest neighbors and all
%  work between the nodes may be pruned.
%  
%  \subsection{Kernel Density Estimation}
%  
%  Kernel Density Estimation (KDE) wraps a small probability density
%  function around each point in some data set in order to estimate that
%  set's distribution.  It is then of interest to determine each point's
%  density in order to detect outliers.  Alternately, we may find
%  densities for a set of queries not from the original data set.
%  \[
%  \map_{q \in Q} \sum_{r \in R} K_h(q,r).
%  \]
%  [[Perhaps mention fitting bandwidth with LOO and L2E, etc.]]
%  
%  \subsection{Nonparametric Bayes Classification}
%  
%  Nonparametric Bayes classification (NBC) applies Bayes' Rule to the
%  results of kernel density estimation in order to predict the class of
%  each of some set of queries $Q$ given sets of references $R_k$ for
%  classes $C_k$, $1 \leq k \leq M$.  The problem is given by
%  \[
%  \map_{1 \leq i \leq |Q|} \argmax_{1 \leq k \leq M} \sum_{1 \leq j \leq |R_k|} K_{h_k}(q_i,r_j),
%  \]
%  where $K(q_i,r_i) \equiv 0$ if $Q = R$, permitting computation for
%  leave-one-out cross-validation.
%  
%  [[Examples of uses.]]
%  
%  Algorithm~\ref{alg:nbc} demonstrate an efficient means of computing
%  NBC in the two-class case.  It forms trees for all involved sets and
%  maintains at query nodes both upper and lower bounds on density
%  contributions from the various classes.  When one class's lower-bound
%  joint probability (found by multiplying the class's lower-bound
%  density and prior) is greater than all other's upper-bounds, then we
%  may safely conclude that all queries within the node should be
%  attributed to that class.
%  
%  \subsection{Multi-radius $n$-point Correlation}
%  
%  The $n$-point correlation is found by counting all unique $n$-tuples
%  of points within some radius $r$ of one another.  It is found with
%  \[
%  \sum_{x_{i_1} \in X} \cdots \sum_{x_{i_n} \in X} I(d(x_{i_j},x_{i_k}) < r \forall j,k),
%  \]
%  where $I(d(x_{i_j},x_{i_k}) < r \forall j,k) \equiv 0$ unless $i_1 <
%  \cdots < i_n$.
%  
%  [[Examples of uses.]]
%  
%  [[Description of algorithm.]]

%%%%%%%%%%%%%%%%%%%%%%%%%%%%%%%%%%%%%%%%%%%%%%%%%%%%%%%%%%%%%%%%%%%%%%%%%%%%%%%
\section{Generalized $N$-body Problems}

The class of Generalized $N$-body Problems (GNPs), or problems
solvable by means similar to those shown above, is a subset of the
problems representable as nested applications of high-order function
reduce.  Reduce is a long-standing and highly versatile feature of
functional programming languages, traditionally operating over a list
from beginning to end or vice versa.  We pose reduce as a function on
unordered input multisets\footnote{Unless otherwise noted, data sets
are understood to be finite multisets, thereby permitting entries with
identical values; alternately, data sets may store a distinct index
for each entry, rendering all entries unique.} rather than lists, and
allow for pre- and postprocessing functions.
\begin{definition}
  A {\bf first-order reduce problem} $\theta$ is a tuple
  $(\mathcal{X},\otimes,f,g)$ of set of possible inputs $\mathcal{X}$,
  commmutative, associative operator $\otimes \colon \mathcal{A}
  \times \mathcal{A} \to \mathcal{A}$, and functions $f \colon
  \mathcal{X} \to \mathcal{A}$ and $g \colon \mathcal{A} \to
  \mathcal{B}$.  Its components form $\Psi_{\theta} = g \comp
  \psi_{\theta}$, where $\psi_{\theta}(X) = \bigotimes_{x \in X} f(x)$
  for nonempty $X \subset \mathcal{X}$.
\end{definition}
\noindent The operators involved in reduce problems form abelian
semigroups; frequently, they also have identities and inverses in
$\mathcal{A}$, thereby forming abelian groups.

\begin{example}
  The expected value of a function under a sampled distribution,
  $\frac{1}{N} \sum_{x \in X} f(x)$, is a first-order reduce problem.
\end{example}

The inner functions of reduce problems may themselves be reduce
problems, for example with $f_1(x) = \Psi_{\theta_2}(X_2)$.  While the
methods presented below can potentially extend to arbitrary nesting
depth, this paper focuses on second-order problems, or problems of two
operators.
\begin{definition}
  A {\bf second-order reduce problem} $\Theta$ is a tuple
  $(\mathcal{X},\mathcal{Y},\otimes,\odot,f,g,h)$ of sets of possible
  inputs $\mathcal{X}$ and $\mathcal{Y}$, commutative, associative
  operators $\otimes \colon \mathcal{A} \times \mathcal{A} \to
  \mathcal{A}$ and $\odot \colon \mathcal{B} \times \mathcal{B} \to
  \mathcal{B}$, and functions $f \colon \mathcal{X} \times \mathcal{Y}
  \to \mathcal{A}$, $g \colon \mathcal{Y} \times \mathcal{A} \to
  \mathcal{B}$, and $h \colon \mathcal{B} \to \mathcal{C}$.  Its
  components form $\Psi_{\Theta} = h \comp \psi_{\Theta}$, where
  $\psi_{\Theta}(Y,X) = \bigodot_{y \in Y} g \left( y,\bigotimes_{x
  \in X} f(x,y) \right)$ for nonempty $X \subset \mathcal{X}$ and $Y
  \subset \mathcal{Y}$.
\end{definition}
\noindent The results of second-order reduce problems may be found
na\"{\i}vely through nested iteration.  If involved functions are
constant time and $X$ and $Y$ are $O(N)$, this method of computation
is $O(N^2)$, which is infeasible for large $N$.

\begin{example}
  The log-likelihood of a mixture of Gaussians, $\sum_{x \in X} \log
  \sum_{k \in C} \omega_k \phi(x | \mu_k, \Sigma_k)$ is a second-order
  reduce problem.
\end{example}

Certain constraints are required by the algorithmic technique
presented in the next section, though we will later introduce a
transform to help work around them.
\begin{definition}
  Second-order reduce problem $\Theta$ is {\bf regular} if $g(y,a) =
  a$ for all $y \in \mathcal{Y}$ and $a \in \mathcal{A}$, and is thus
  given by $\Psi_{\Theta} = h \comp \psi_{\Theta}$, where
  $\psi_{\Theta}(Y,X) = \bigodot_{y \in Y} \bigotimes_{x \in X}
  f(x,y)$.  Note that $\mathcal{B} = \mathcal{A}$.
\end{definition}
\begin{definition}
  Regular second-order reduce problem $\Theta$ is {\bf block
  decomposable} if, for all nonempty partitions $X^{\!L} \cup X^{\!R}
  = X \subset \mathcal{X}$ and nonempty $Y \in \mathcal{Y}$,
  $\GNP(Y,X) = \GNP(Y,X^{\!L}) \otimes \GNP(Y,X^{\!R})$.  Such a
  problem\footnote{Observe that commutativity and associativity ensure
  that $\GNP(Y,X) = \GNP(Y^{\!L},X) \odot \GNP(Y^{\!R},X)$.} is known
  as a {\bf second-order generalized $N$-body problem}.
\end{definition}

% \subsection{The Map Operator}

{\bf The Map Operator.}  High-order function map is suitable for all
problems that compute separate results for some set of queries and may
help with some problems that do not.  Reduce problems subsume the
functionality of map with $\map_{x \in X} f(x) \equiv \bigcup_{x \in
X} \{(x,f(x))\} = \{(x,f(x)) | x \in X\}$.  We distribute the
formation of key-value pairs in order to form regular reduce problems.
%Talk about distibution
\begin{lemma}
  Second-order reduce problem $\Theta$ with $\odot = \map$ is
  equivalent to $\overrightarrow{\Theta}$ with
  \[ \begin{array}{rclrcl}
    \overrightarrow{f}(x,y) & = & \{(y, f(x,y))\}, & \overrightarrow{g}(A) & = & \{(y, g(y,v)) | (y,v) \in A\}, \\
    A \otimesvec B & = & \{(y, u \otimes v) | (y,u) \in A, (y,v) \in B\}, & A \odotvec B & = & A \cup B.
  \end{array} \]
\end{lemma}
\begin{proof}
  During na\"{\i}ve computation, arguments presented to $\otimesvec$
  and $\overrightarrow{g}$ are singleton sets for some $y \in Y$.
  Vector operations then trivially match the original version of the
  algorithm.
\end{proof}
\noindent Vectorized operations become more interesting after the
application of block decomposition.

% \subsection{Block Decomposability}

{\bf Block Decomposable Problems.} The primary advantage of GNPs is
the significant ability to rearrange their order of computation.  The
na\"{\i}ve computation of a second-order GNP forms a grid of
evaluations of $f$ connected with $\otimes$ and $\odot$ as shown in
the left half of Figure~\ref{fig:grid}.  A single application of block
decomposition might result in the right half of Figure~\ref{fig:grid}.
Further applications can form any hierarchy of nested rectangular
regions, possibly with permuted orderings of rows or columns.

\begin{figure}
  \begin{eqnarray*}
    \begin{array}{ccccccccc}
      \scriptstyle ( \!\!\!&\scriptstyle\!\!\! f(x_1,y_1) \!\!\!&\scriptstyle\!\!\! \otimes \!\!\!&\scriptstyle\!\!\! f(x_2,y_1) \!\!\!&\scriptstyle\!\!\! \otimes \!\!\!&\scriptstyle\!\!\! \cdots \!\!\!&\scriptstyle\!\!\! \otimes \!\!\!&\scriptstyle\!\!\! f(x_N,y_1) \!\!\!&\scriptstyle\!\!\! ) \\
      \multicolumn{9}{c}{\scriptstyle \odot} \\
      \scriptstyle ( \!\!\!&\scriptstyle\!\!\! f(x_1,y_2) \!\!\!&\scriptstyle\!\!\! \otimes \!\!\!&\scriptstyle\!\!\! f(x_2,y_2) \!\!\!&\scriptstyle\!\!\! \otimes \!\!\!&\scriptstyle\!\!\! \cdots \!\!\!&\scriptstyle\!\!\! \otimes \!\!\!&\scriptstyle\!\!\! f(x_N,y_2) \!\!\!&\scriptstyle\!\!\! ) \\
      \multicolumn{9}{c}{\scriptstyle \odot} \\
      \multicolumn{9}{c}{\scriptstyle \vdots} \\
      \multicolumn{9}{c}{\scriptstyle \odot} \\
      \scriptstyle ( \!\!\!&\scriptstyle\!\!\! f(x_1,y_M) \!\!\!&\scriptstyle\!\!\! \otimes \!\!\!&\scriptstyle\!\!\! f(x_2,y_M) \!\!\!&\scriptstyle\!\!\! \otimes \!\!\!&\scriptstyle\!\!\! \cdots \!\!\!&\scriptstyle\!\!\! \otimes \!\!\!&\scriptstyle\!\!\! f(x_N,y_M) \!\!\!&\scriptstyle\!\!\! )
    \end{array}
    & = &
    \begin{array}{ccc}
      \left( \begin{array}{c}
	\scriptstyle \!\!\!f(x_1,y_1)\!\!\! \\
	\scriptstyle \odot \\
	\scriptstyle \!\!\!f(x_1,y_2)\!\!\! \\
	\scriptstyle \odot \\
	\scriptstyle \vdots \\
	\scriptstyle \odot \\
	\scriptstyle \!\!\!f(x_1,y_M)\!\!\!
      \end{array} \right)
      \!\!\!&\scriptstyle\!\!\! \otimes \!\!\!&\!\!\!
      \left( \begin{array}{ccccccc}
	\scriptstyle ( \!\!\!&\scriptstyle\!\!\! f(x_2,y_1) \!\!\!&\scriptstyle\!\!\! \otimes \!\!\!&\scriptstyle\!\!\! \cdots \!\!\!&\scriptstyle\!\!\! \otimes \!\!\!&\scriptstyle\!\!\! f(x_N,y_1) \!\!\!&\scriptstyle\!\!\! ) \\
	\multicolumn{7}{c}{\scriptstyle \odot} \\
	\scriptstyle ( \!\!\!&\scriptstyle\!\!\! f(x_2,y_2) \!\!\!&\scriptstyle\!\!\! \otimes \!\!\!&\scriptstyle\!\!\! \cdots \!\!\!&\scriptstyle\!\!\! \otimes \!\!\!&\scriptstyle\!\!\! f(x_N,y_2) \!\!\!&\scriptstyle\!\!\! ) \\
	\multicolumn{7}{c}{\scriptstyle \odot} \\
	\multicolumn{7}{c}{\scriptstyle \vdots} \\
	\multicolumn{7}{c}{\scriptstyle \odot} \\
	\scriptstyle ( \!\!\!&\scriptstyle\!\!\! f(x_2,y_M) \!\!\!&\scriptstyle\!\!\! \otimes \!\!\!&\scriptstyle\!\!\! \cdots \!\!\!&\scriptstyle\!\!\! \otimes \!\!\!&\scriptstyle\!\!\! f(x_N,y_M) \!\!\!&\scriptstyle\!\!\! )
      \end{array} \right)
    \end{array}
  \end{eqnarray*}
  \caption{\label{fig:grid}{\em (Left)} Na\"{i}ve computation of a
  second-order GNP.  {\em (Right)} An application of block
  decomposition restructures computation.}
\end{figure}

While $\mathcal{X}$, $\mathcal{Y}$, and $f$ may have some effect on
whether a problem is block decomposable, the most significant players
are $\otimes$ and $\odot$.  There are two important classes of
operator pairs that gaurantee block decomposability.
\begin{lemma}\label{lem:self}
  A regular second-order reduce problem is block decomposable if
  $\odot = \otimes$.
\end{lemma}
\begin{proof}
  By commutativity and associativity, we may rearrange
  \[ \begin{array}{ll}
    \multicolumn{2}{l}{\displaystyle \GNP(Y,X) = \bigotimes_{y \in Y} \bigotimes_{x \in X} f(x,y) = \bigotimes_{x \in X} \bigotimes_{y \in Y} f(x,y) = \bigotimes_{x \in X^{\!L}} \bigotimes_{y \in Y} f(x,y) \otimes \bigotimes_{x \in X^{\!R}} \bigotimes_{y \in Y} f(x,y)} \\
    & \displaystyle = \bigotimes_{y \in Y} \bigotimes_{x \in X^{\!L}} f(x,y) \otimes \bigotimes_{y \in Y} \bigotimes_{x \in X^{\!R}} f(x,y) = \GNP(Y,X^{\!L}) \otimes \GNP(Y,X^{\!R}).
  \end{array} \]
\end{proof}

\begin{example}[The 2-point Correlation]
  By Lemma~\ref{lem:self}, $\sum_{y \in X} \sum_{x \in X} I(d(x,y)
  \leq r)$ is a GNP.
\end{example}

\begin{lemma}\label{lem:map}
  A regular second-order reduce problem is block decomposable if
  $\odot = \map$.
\end{lemma}
\begin{proof}
  For $\GNP(Y,X) = \map_{y \in Y} \bigotimes_{x \in X} f(x,y) \equiv
  \bigcup_{y \in Y} \bigotimesvec_{x \in X} \{(y,f(x,y))\} =
  \GNPvec(Y,X)$ and by commutativity, associativity, and the
  definition of map, we have
  \[ \begin{array}{ll}
    \multicolumn{2}{l}{\displaystyle \GNPvec(Y,X) \equiv \Big\{ \!\Big( y,\bigotimes_{x \in X} f(x,y) \Big)\! \Big| y \in Y \Big\} = \Big\{ \!\Big( y,\bigotimes_{x \in X^{\!L}} f(x,y) \otimes \bigotimes_{x \in X^{\!R}} f(x,y) \Big)\! \Big| y \in Y \Big\}} \\
    & \displaystyle = \Big\{ \!\Big( y,\bigotimes_{x \in X^{\!L}} f(x,y) \Big)\! \Big| y \in Y \Big\} \otimesvec \Big\{ \!\Big( y,\bigotimes_{x \in X^{\!R}} f(x,y) \Big)\! \Big| y \in Y \Big\} \equiv \GNPvec(Y,X^{\!L}) \otimesvec \GNPvec(Y,X^{\!R}).
  \end{array} \]
\end{proof}

\begin{example}[All-nearest-neighbors]
  By Lemma~\ref{lem:map}, $\map_{q \in Q} \argmin_{r \in R} d(q,r)$ is
  a GNP.
\end{example}

% \subsection{Transforming Problems into GNPs}

{\bf Transforming Problems into GNPs.}  All second-order reduce
problems that are not GNPs may be transformed into GNPs.  Simple
algebra can often distribute or factor non-identity $g$ into less
obstructive positions, but failing that (or failing block
decomposability if $g$ is not a problem), we may form a GNP by
injecting map.
\begin{lemma}
  Second-order reduce problem $\Theta$ is equivalent to GNP
  $\breve{\Theta}$ with
  \[
  \Psi_{\Theta}(Y,X) = h \Big( \bigodot_{y \in Y} g \Big( \bigotimes_{x \in X} f(x,y) \Big) \Big) \equiv h \comp {\textstyle \bigodot} \comp \overrightarrow{g} \Big( \map_{y \in Y} \bigotimes_{x \in X} f(x,y) \Big) = \Psi_{\breve{\Theta}}(Y,X),
  \]
  where unsubscripted $\bigodot$ is understood to reduce the values in
  a set of key-value pairs.
\end{lemma}
\begin{proof}
  Observe that $\bigodot_{x \in X} f(x) = \bigodot{} \map_{x \in X}
  f(x)$ for arbitrary $f$ and $\bigodot$.  Further, observe that
  $\map_{x \in X} g(x,f(x)) = \overrightarrow{g} \left( \map_{x \in X}
  f(x) \right)$ for arbitrary $f$ and $g$.  Applying these in
  succession to the left-hand side of the above demonstrates its
  equality to the right.  By Lemma~\ref{lem:map}, $\breve{\Theta}$ is
  a GNP.
\end{proof}

\begin{example}[Nonparametric Bayes Classification]
\end{example}

%%%%%%%%%%%%%%%%%%%%%%%%%%%%%%%%%%%%%%%%%%%%%%%%%%%%%%%%%%%%%%%%%%%%%%%%%%%%%%%
\section{The Generalized $N$-body Algorithm}

The block decomposable nature of GNPs suggests a recursive alternative
to the na\"{\i}ve computation,
\[
\GNP(Y,X) = \left\{ \begin{array}{lrr}
  f(x,y) & \multicolumn{2}{r}{\mbox{\rm if } X = \{x\} \mbox{ \rm and } Y = \{y\},} \\
  \multicolumn{2}{l}{\GNP(Y,X^{\!L}) \otimes \GNP(Y,X^{\!R})} & \mbox{\rm if } X \succ Y, \\
  \multicolumn{2}{l}{\GNP(Y^{\!L}\!,X) \odot \GNP(Y^{\!R}\!,X)} & \mbox{\rm otherwise},
\end{array} \right.
\]
where $X \succ Y$ is some means of deciding what to split first, such
as $|X| \geq |Y|$.  Recursion forms a binary tree with one leaf per
element of $X \times Y$.  Exhuastive computation thus requires time
$O(N^2)$, the same as the na\"{\i}ve algorithm.  It may be possible,
however, to obtain results for some components of the recursive block
decomposition without computing them exhaustively (i.e.~to {\em prune}
them).  Exploiting this, we hope to drive the expected running time
down.

% \subsection{Summaries and Statistics}

{\bf Summaries Results.}  In order to accelerate computation, we need
some quick means of summarizing the possible results of $\GNP(Y,X)$.
This typically involves consideration of $X$ and $Y$ at the abstract
level formed by a concise (and ideally precomputed) set of statistics
on the two.
\begin{definition}
  For descriptive statistics $\sigma_x \colon 2^{\mathcal{X}} \to
  \mathcal{S}_x$ and $\sigma_y \colon 2^{\mathcal{Y}} \to
  \mathcal{S}_y$, let {\bf summary results} $\GNP[\sigma] \colon
  \mathcal{S}_y \times \mathcal{S}_x \to 2^\mathcal{A}$ be such that
  \[
  \GNP[\sigma](\sigma_y(Y),\sigma_x(X)) \supseteq \{\GNP(Y',X') | X' \st \sigma(X') = \sigma(X), Y' \st \sigma(Y') = \sigma(Y)\}.
  \]
\end{definition}
\noindent Intuitively, $\GNP[\sigma](\sigma_y(Y),\sigma_x(X))$
represents all possible results of $\GNP(Y,X)$ given what we know
about $X$ and $Y$.  It is permitted to be a superset of such results
because the exact set may be costly or impossible to represent.  Note
that functions $\sigma_x$, $\sigma_y$, and $\GNP[\sigma]$ are not
unique for a given GNP; indeed, chosing the right statistics can
significantly impact running time.  A common example of statistics is
finding bounding boxes of data in Euclidean space.  Summaries computed
from these may be represented with upper and lower bounds on distances
between points in $X$ and $Y$.

% \subsection{Intrinsic Pruning}

Summarization leads directly to our first form of pruning.
\begin{lemma}[Intrinsic Pruning]
  We may prune when summary results form a singleton set.
\end{lemma}
\begin{proof}
  Setting $X' = X$ and $Y' = Y$ meets the requirements for inclusion
  in $\GNP[\sigma](\sigma_y(Y),\sigma_x(X))$.  Thus, summary results
  contain the exact result.  Thus, singleton
  $\GNP[\sigma](\sigma_y(Y),\sigma_x(X)) = \{\GNP(Y,X)\}$ and we are
  free to shortcut all further computation on $\GNP(Y,X)$ with this
  value.
\end{proof}

\begin{example}[Pruning in the 2-point Correlation]
  We may perform intrinsic pruning when considering regions shown to
  be completely inside or outside radius $r$ of one another.
\end{example}
\begin{proof}
  Let $\sigma_x$ find bounding boxes for points in $\mathcal{X}$ and
  let $\disthrectmax$ and $\disthrectmin$ find upper and lower bound
  distances between bounding boxes.  For $x \in X$ and $y \in Y$,
  bounds on $I(d(x,y) \leq r)$ are then $\left[
  I(\disthrectmax(\sigma_x(X),\sigma_x(Y)) \leq r),
  I(\disthrectmin(\sigma_x(X),\sigma_x(Y)) \leq r) \right]$.  We may thus
  define
  \[
  \GNP[\sigma](Y,X) = \left[ |Y| \cdot |X| \cdot I(\disthrectmax(\sigma_x(X),\sigma_x(Y)) \leq r), |Y| \cdot |X| \cdot I(\disthrectmin(\sigma_x(X),\sigma_x(Y)) \leq r) \right].
  \]
  This set is $\{|Y| \cdot |X|\}$ when
  $D^{\!U}(\sigma_x(X),\sigma_x(Y)) \leq r$ and $\{0\}$ when
  $D^{\!L}(\sigma_x(X),\sigma_x(Y)) > r$.
\end{proof}

% \subsection{Iterative Refinement.}

{\bf Iterative Refinement.}  Further pruning is possible in some
problems by considering results gathered from other parts of
computation.  This is assisted by the ability to compose summary
results.
\begin{definition}
  Let {\bf summary composition operators} $\otimeshat, \odothat \colon
  2^{\mathcal{A}} \times 2^{\mathcal{A}} \to 2^{\mathcal{A}}$ be such
  that
  \[ \begin{array}{rcl}
    A \otimeshat B \supseteq \{a \otimes b | a \in A, b \in B\} & \mbox{ and, likewise, } & A \odothat B \supseteq \{a \odot b | a \in A, b \in B\}
  \end{array} \]
  for summary results $A,B \subset \mathcal{A}$.  For singleton $A$
  and $B$, we require $A \otimeshat B$ and $A \odothat B$ to be
  singleton.
\end{definition}
\noindent We develope a notion of itertive refinement by means of
replacing summary result sets for the various components of
computation with composed summary results for their left and right
subcomponents.
\begin{definition}
  {\bf Iterative refinement} constructs a binary tree wherein each
  node $\GNP[\Sigma](Y,X)$ represents composed summary results for a
  component of computation introduced via block decomposition.  First,
  initialize $\GNP[\Sigma](Y_{root},X_{root}) \gets
  \GNP[\sigma](\sigma_y(Y_{root}),\sigma_x(X_{root}))$.  Then,
  repeatedly select some node $\GNP[\Sigma](Y,X) =
  \GNP[\sigma](\sigma_y(Y),\sigma_x(X))$ and replace it with
  \[
  \GNP[\Sigma](Y,X) \gets \left\{ \begin{array}{lrr}
    \{f(x,y)\} & \multicolumn{2}{r}{\mbox{\rm if } X = \{x\} \mbox{ \rm and } Y = \{y\}} \\
    \multicolumn{2}{l}{\GNP[\Sigma](Y,X^{\!L}) \otimeshat \GNP[\Sigma](Y,X^{\!R})} & \mbox{\rm if } X \succ Y, \\
    \multicolumn{2}{l}{\GNP[\Sigma](Y^{\!L}\!,X) \odothat \GNP[\Sigma](Y^{\!R}\!,X)} & \mbox{\rm otherwise},
  \end{array} \right.
  \]
  where newly introduced child nodes are initialized
  $\GNP[\Sigma](Y',X') \gets \GNP[\sigma](\sigma_y(Y'),\sigma_x(X')$.
  The value of each ancestor $\GNP[\Sigma](Y,X)$ is understood to
  reflect changes made to its descendents.
\end{definition}
\noindent Iterative refinement adds one node to the tree per
replacement, and thus must terminate after $O(N^2)$ steps.  It is
useful to navigate this tree, with functions $\leftchild$,
$\rightchild$, $\parent$, and $\sibling$ defined intuitively.
Further, let $\op(\GNP[\Sigma](Y,X))$ return $\otimes$ or $\odot$
depending on $X \succ Y$.

Refinement need not be performed in any particular pattern.
Depth-first is often a good choice due to its low overhead, though
pruning in some problems strongly favors other expansion patterns.

% \subsection{Extrinsic Pruning}

Iterative refinement allows us to prune components when more precise
knowledge of their results cannot affect the global result.
\begin{lemma}[Extrinsic Pruning]
  For node $\GNP[\Sigma](Y,X)$ at depth $D$ of tree
  $\GNP[\Sigma](Y_{root},X_{root})$ and path $A_0,\ldots,A_D$ given by
  $A_D = \GNP[\Sigma](Y,X)$ and $A_{d-1} = parent(A_{d})$ for $1 \leq
  d \leq D$, we may prune if
  \[
  \forall a_1 \in \sibling(A_1),\ldots,a_D \in \sibling(A_D)~~ \exists r \st~ \forall b \in \GNP[\Sigma](Y,X)~~ s_0 = r,
  \]
  where $s_D = b$ and $s_{d-1} = a_d \mathbin{\op(A_{d-1})} s_d$ for $1 \leq d
    \leq D$.
\end{lemma}
\begin{proof}
  Because $\GNP[\Sigma](Y,X)$ containts the exact result for the
  represented region of computation, the logical condition given above
  implies that all $b \in GNP[\Sigma](Y,X)$ yield the same final
  result as $\GNP(Y,X)$, regardless of the results of the rest of the
  problem.  Thus, we are free to shortcut all further computation on
  $\GNP(Y,X)$ with any $b \in \GNP[\Sigma](Y,X)$.
\end{proof}

\begin{example}[Pruning in All-nearest-neighbors]
  We may perform extrinsic pruning when references are shown to be
  farther away than queries' nearest-neighbors found so far.
\end{example}
\begin{proof}
  Again, let $\sigma_x$ find bounding boxes for points in
  $\mathcal{X}$ and let $\disthrectmax$ and $\disthrectmin$ find upper and
  lower bound distances between bounding boxes.  We represent summary
  results with
  \begin{eqnarray*}
    \GNP[\sigma](\sigma_x(Q),\sigma_x(R)) & = & \eta(\{(q,[\disthrectmin(\sigma_x(Q),\sigma_x(R)), \disthrectmax(\sigma_x(Q),\sigma_x(R))]) | q \in Q\}), \\
    \eta(A) \otimeshat \eta(B) & = & \eta(\{(q,[\min(l_a,l_b),\min(u_a,u_b))]) | (q,[l_a,u_a]) \in A, (q,[l_b,u_b]) \in B\}), \\
    \eta(A) \odothat \eta(B) & = & \eta(A \cup B),
  \end{eqnarray*}
  where $\eta$ intuitively converts\footnote{We only need $\eta$
  because the most convenient representation of All-NN's summary
  results is not the same type as is needed by the generalized math.}
  a set of key-range pairs into the set of all sets of key-value pairs
  with values within their corresponding ranges.  Due to properties of
  $\min$ and $\map$,
%  Because $\min(v,v) = v$ and because, for $Q' \subseteq Q^{\!L}$,
%  \[
%  \overrightarrow{\min}(\GNP(\kdleft{Q},R) \cup \GNP(\kdright{Q},R),\GNP(Q',R)) = \overrightarrow{\min}(\GNP(\kdleft{Q},R),\GNP(Q',R'))
%  \]
%  we have $s_0 = \overrightarrow{\min}(s_0,b)$.  Because $s_0$
%  represents a final result,
  \[
  \forall a \in \GNP[\Sigma](Y_{root},X_{root})~~ \exists r \st~ \forall b \in \GNP[\Sigma](Y,X)~~ \overrightarrow{\min}(a,b) = r.
  \]
  is equivalent\footnote{This claim deserves a proof, but we omit it
  for brevity.} to the extrinsic prune test.  If, for all matching
  $(q,v_a) \in a$ and $(q,v_b) \in b$, we have $v_a < v_b$, then
  $\overrightarrow{\min}(a,b) = a$.  Thus, if for all matching
  $(q,[l_a,u_a]) \in A$ and $(q,[l_b,u_b]) \in B$, with $\eta(A) =
  \GNP[\Sigma](Q_{root},R_{root})$ and $\eta(B) = \GNP[\Sigma](Q,R)$,
  we have $u_a \leq l_b$, then we may always choose $r = a$.  In
  words, we may prune when the lower bound distance between the
  queries and references is greater than the greatest of the queries'
  candidate nearest-neighbor distance found so far.
\end{proof}

% \subsection{Thresholded Pruning}

Postprocessing function $h$ introduces a
third form of pruning.
\begin{lemma}[Thresholded Pruning]
  We may prune if $\{h(b) | b \in B\}$ is singleton for all $B \in
  S_0$.
\end{lemma}
\begin{proof}
  Similar to extrinsic pruning.
\end{proof}

\begin{example}[Pruning in Nonparametric Bayes Classification]
  We may perform thresholded pruning when bounds on queries' kernel
  sums restrict their values to be positive, negative, or zero.
\end{example}
\begin{proof}
  Yet again, let $\sigma_x$ find bounding boxes for points in
  $\mathcal{X}$ and let $D^{\!U}$ and $D^{\!L}$ find upper and lower
  bound distances between bounding boxes.  We represent summary
  results with
  \begin{eqnarray*}
    \GNP[\sigma](\sigma_x(Q),\sigma_x(R)) & = & \eta \left( \left\{ \left.\! \left( q, \left[ \begin{array}{ll}
	\lefteqn{|R^+| \cdot \pi^+ K(D^{\!U}(\sigma_x(Q),\sigma_x(R)))} \\
	& \mbox{} + |R^-| \cdot \pi^- K(D^{\!L}(\sigma_x(Q),\sigma_x(R))), \\
	\lefteqn{|R^+| \cdot \pi^+ K(D^{\!L}(\sigma_x(Q),\sigma_x(R)))} \\
	& \mbox{} + |R^-| \cdot \pi^- K(D^{\!U}(\sigma_x(Q),\sigma_x(R)))
    \end{array} \right] \right) \right| q \in Q \right\} \right) , \\
    \eta(A) \otimeshat \eta(B) & = & \eta(\{(q,[l_a + l_b,u_a + u_b)]) | (q,[l_a,u_a]) \in A, (q,[l_b,u_b]) \in B\}), \\
    \eta(A) \odothat \eta(B) & = & \eta(A \cup B),
  \end{eqnarray*}
  where $\eta$ converts types as before.  We may reason about $S_0$
  directly from $\GNP[\Sigma](Q_{root},R_{root})$.  If, for all
  $(q,[l,u]) \in A$, with $P(A) = \GNP[\Sigma](Q_{root},R_{root})$ and
  $q \in Q$, we have $l > 0$, $u < 0$, or $l = u = 0$, then no $p \in
  \GNP[\Sigma](Q,R)$ can yield distinct results of postprocessing
  function $\sign$ regardless of the valuation of all other components
  of computation.  In words, when bounds on the kernel sums for some
  selection of queries restrict results to be positive, negative, or
  zero, we may prune further computation on those queries for all
  remaining references.
\end{proof}

% \subsection{Approximation Pruning}

% {\bf Approximation Pruning.}  Some problems do not lend themselves to
% any of the above forms of pruning.  For these, we may still be able to
% find approximate results with bounded error more quickly than
% exhaustive computation.  We must first establish a notion of error.
% \begin{definition}
%   Let $div \colon 2^\mathcal{A} \to \mathcal{A} \times \mathcal{A} \to
%   \mathbb{R}$ be some measure of divergence between results given the
%   summary results of the full computation, written $div(a,b |
%   \GNP[\Sigma](Y,X))$.  Define $err \colon 2^\mathcal{A} \to
%   2^\mathcal{A} \to \mathbb{R}$ to be $err(\GNP[\Sigma](Y',X') |
%   \GNP[\Sigma](Y,X)) = \min_{\widehat{a} \in \mathcal{A}} \max_{b \in
%   \GNP[\Sigma](Y',X')} div(\widehat{a},b | \GNP[\Sigma](Y,X))$.
% \end{definition}
% \noindent For example, we might have $div(a,b | \GNP[\Sigma](Y,X)) =
% |a - b| / \min_{c \in \GNP[\Sigma](Y,X)} |c|$, or relative error.
% 
% Given a desired $\epsilon$, iterative refinement may terminate once
% $err(\GNP[\Sigma](Y,X) | \GNP[\Sigma](Y,Z)) < \epsilon$, returning the
% minimizing $\widehat{a}$ found for $err$.  A simple algorithm might
% then check the error of $\GNP[\Sigma](Y,X)$ after each step, refining
% components in the order of descending error in attempt to make the
% most of its work.  This approach must use a priority queue to manage
% expansion, incuring significant overhead.
% 
% An alternate approach distributes error to the various components of
% computation.
% \begin{lemma}
%   Given summary results $A, B \subset \mathcal{A}$, let operators
%   $\otimestilde, \odottilde \colon \mathbb{R} \times \mathbb{R}
%   \to \mathbb{R}$ be such that
%   \begin{eqnarray*}
%     err(A | \GNP[\Sigma](Y,X)) \otimestilde err(B | \GNP[\Sigma](Y,X)) & = & err(A \otimeshat B | \GNP[\Sigma](Y,X)) \\
%     err(A | \GNP[\Sigma](Y,X)) \odottilde err(B | \GNP[\Sigma](Y,X)) & = & err(A \odothat B | \GNP[\Sigma](Y,X)).
%   \end{eqnarray*}
%   Given some desired $\epsilon$, let $\epsilon^*$ be such that
%   $\bigodottilde_{y \in Y} \bigotimestilde_{x \in X} \epsilon^* =
%   \epsilon$ and let $\epsilon' = \bigodottilde_{y \in Y'}
%   \bigotimestilde_{x \in X'} \epsilon^*$.  We may perform {\em
%   approximation pruning} when $err(\GNP[\Sigma](Y',X') |
%   \GNP[\Sigma](Y,X)) < \epsilon'$.
% \end{lemma}

% \subsection{Practical Considerations: Trees, Bounds}

\section{Derivation of Example Algorithms}

\section{Affinity Propagation}

%--------- Affinity Propagation Starts

Affinity propagation \cite{affinity} is a recent clustering technique
that chooses {\em exemplars} from the data such in attempt to maximize
the sum of similarities between all points and their nearest exemplar.
Two points $x_i$ and $x_j$ have similarity $\simil{i}{j}$, and special
case $\simil{i}{i}$ is set to parameter $p$, the {\em preference} of
points to be exemplars.  The number of clusters to find is not
explicitly specified, but is positively correlated with $p$.

The algorithm alternateingly updates messages $\respo{i}{j}$ and
$\avail{i}{j}$ with
\[ \begin{array}{rcl}
  \respo{i}{j} \gets \simil{i}{j} - \max_{j' \neq j} (\avail{i}{j'} + \simil{i}{j'})
  & \text{ and } &
  \avail{i}{j} \gets \cneg{i}{j}( \sum_{i' \neq i} ( \cpos{i'}{j}(\respo{i'}{j}) ) )
\end{array} \]
\noindent where $\cpos{i}{j}(x) = \max(I(i = j)x,x)$ clamps $x$ to be non-negative when $i \neq j$ and $\cneg{i}{j}$

\noindent where $\cpos{i}{j}(x)$ equals $x$ when $i=j$, and $\max(x,0)$ otherwise; likewise, $\cneg{i}{j}$ for $\min(x, 0)$.
Trivially, $\respo{}{}$ and $\avail{}{}$ are $N \times N$ matrices, where $N$ is the number of points.
To create a sub-quadratic GNP, we define vectors $\vecalpha$ and $\vecrho$, which we compute instead:
\begin{equation*}
  \begin{array}{lcr}
    \falphaj{i}{j} \gets \min_{j' \neq j} (-\avail{i}{j'} - \simil{i}{j'})
    &\quad&
    \frho{j} \gets \sum_{i'} ( \cpos{i'}{j}(\respo{i'}{j}) )
  \end{array}
\end{equation*}

\noindent Note $\falphaj{i}{j}$ depends on $j$ only for exclusion from $\min$; computationally, one finds the first and second minimum, and chooses the first or second depending on $j$.
Substituting, $\respo{}{}$ and $\avail{}{}$ are:
\begin{equation*}
  \begin{array}{lcccr}
    \respo{i}{j} = \simil{i}{j} + \falphaj{i}{j}
    &\quad&
    \avail{i}{j} = \cneg{i}{j} (\frho{j} - \cpos{i}{j}(\respo{i}{j}) )
  \end{array}
\end{equation*}

\noindent 
A bit of algebraic substitution defines $\vecalpha$ and $\vecrho$ in terms of each other:
\begin{equation*}
  \begin{array}{lr}
    \vecalpha \gets \map_{i} \argmin^2_{j} \left( \cpos{i}{j}(\cpos{i}{j}(\simil{i}{j} + \falphaj{i}{j}) - \frho{j}) - \simil{i}{j} \right)
    &~
    \vecrho \gets \map_{j} \sum_{i} \left( \cpos{i}{j}(\simil{i}{j} + \falphaj{i}{j}) \right)
  \end{array}
\end{equation*}

We now have a problem nearly\footnote{To aid convergence, affinity propagation adds noise to the similarity matrix and dampens both $\respo{}{}$ and $\avail{}{}$ between time-steps.  Since we store no matrices, we both dampen and add noise to $\vecrho$.} indentical to traditional affinity propagation.
However, we can show that this is a generalized $N$-body problem.

{\bf Proposition}.
The computations of both $\vecrho$ and $\vecalpha$ are generalized $N$-body problems.

{\bf Proof}.
Ryan, prove this please.

We may now develop a dual-tree algorithm.
First, we build a tree on the data points, assuming they are in a metric space\footnote{Euclidean distance over real vectors is the example we implemented.}.
Notationally, we denote $Q'$ a subset or node of points that are being $\map$ed over, and $R'$ as a subset or node of points for the inner operator.

Stub: For $\vecrho$, there is a simple intrinsic prune.
We can bound the contribution to the summation as follows:
$$
  \begin{array}{rcl}
    \sigma_{\rho}(Q,R) &=& \minmax_{j \in Q} \sum_{i \in R} ( \cpos{i}{j}(\simil{i}{j} + \falphaj{i}{j}) )
    \\ &=& [|R|(S^l(\sigma(Q),\sigma(R)) + \falphamin(\sigma(R))),
            |R|c^{+}(S^u(\sigma(Q),\sigma(R)) + \falphamax(\sigma(R)))]
  \end{array}
$$

\noindent where $\falphamin$ is the least first minimum $\alpha$, $\falphamax$ is the greatest second minimum $\alpha$, $S^l$ and $S^u$ are the bounds on the similarity matrix.
Both $\falphamin$ and $\falphamax$ can be computed in a linear-time bottom-up pass, whereas $S^l$ and $S^u$ computed using bounding boxes, and parameter $p$ if the nodes overlap.

Stub: For $\vecalpha$:
$$
  \begin{array}{rcl}
    \sigma_{\alpha}(Q,R) &=& \minmax_{i \in Q} \min^2_{j \in R} ( \cpos{i}{j}(\cpos{i}{j}(\simil{i}{j} + \falphaj{i}{j}) - \frho{j}) - \simil{i}{j} )
    \\
        &=& [
        \falphamin(\sigma(Q)) + \frhomin(\sigma(R)),
    \\ && \text{  }
          c^{+}(c^{+}(S^u(\sigma(Q), \sigma(R)) + \falphamax(\sigma(Q))) - \frhomin(\sigma(R))) - \falphamax(\sigma(Q))
        ]
  \end{array}
$$

\begin{figure}
  \begin{equation*}
    \begin{array}{ll}
      \begin{array}{l}
        \text{init all nodes }Q \subseteq \kdroot{Q}\text{: }\falphamax{'}(Q) \gets \infty
        \\ \text{init all points }q \in \kdroot{Q}\text{: }\falphaj{q}{1\text{ and }2}' \gets (\emptyset, \infty)
        %\\ \text{alpha}(\kdroot{Q}, \kdroot{R})
        %\\
        \\ \text{let }\falphamin_{\text{cand}}(Q, R) = \falphamin(\sigma(Q)) + \frhomin(\sigma(R))
        \\ \text{procedure alpha}(Q,R)\text{:}
        \x \text{if }\falphamax{'}(Q) < \falphamin_{\text{cand}}(Q, R)\text{: return}
        \x \text{elif }Q = \{q\} \text{ and } R = \{r\}
        \xx \alphacand \gets \cpos{q}{r}(\cpos{q}{r}(\simil{q}{r} + \falphaj{q}{r}) - \frho{r}) - \simil{q}{r}
        \xx \text{if }\alphacand < \falphaj{q}{1}'\text{: } \falphaj{q}{2}' \!\gets\! \falphaj{q}{1}'; \falphaj{q}{1} \!\gets\! (r, \alphacand))
        \xx \text{elif }\alphacand < \falphaj{q}{2}'\text{: } \falphaj{q}{2}' \!\gets\! (r, \alphacand)
        \xx \falphamax{'}(\{q\}) \gets \falphaj{q}{2}'
        \x \text{elif }|Q| \geq |R|\text{:}
        \xx \text{alpha}(\kdleft{Q}, R); \text{ alpha}(\kdright{Q}, R)
        \xx \falphamax{'}(Q) \gets \max(\falphamax{'}(\kdleft{Q}), \falphamax{'}(\kdright{Q}))
        \x \text{else prioritize by min }\falphamin_{\text{cand}}(Q,\{\kdleft{R}, \kdright{R}\}){:}
        \xx \text{alpha}(Q, \kdleft{R}); \text{ alpha}(Q, \kdright{R})
      \end{array}
      &
      \begin{array}{l}
        \text{init all points }q \in \kdroot{Q}\text{: }
        \x \frho{q}'\gets c^{-}(\simil{q}{q} + \falphaj{q}{q})
        %\\ \text{rho}(\kdroot{Q}, \kdroot{R})
        %\\
        \\ \text{procedure rho}(Q,R)\text{:}
        \x \text{if }S^u(\sigma(Q),\sigma(R)) + \falphamax(\sigma(R)) \leq 0\text{:}
        \xx \text{return}
        \x \text{elif }Q = \{q\} \text{ and } R = \{r\}
        \xx \frho{q}' \gets \frho{q}' + c^{+}(\simil{r}{q} + \falphaj{r}{q})
        \x \text{elif }|Q| \geq |R|\text{:}
        \xx \text{rho}(\kdleft{Q}, R); \text{ rho}(\kdright{Q}, R)
        \x \text{else:}
        \xx \text{rho}(Q, \kdleft{R}); \text{ rho}(Q, \kdright{R})
      \end{array}
    \end{array}
  \end{equation*}
  \label{fig:alpharho}
\end{figure}

{\bf Speed.}

We implemented the above algorithms for computing $\vecalpha$ and $\vecrho$ in C++ and compare to the C code provided by Frey and Dueck.


We compare our algorithm to the C code provided by Frey and Dueck\cite{frey_dueck_code}.
Their code implements 

\fig{affinity_speed}.


 for even a million points, two trillion single-precision floating-point numbers are required -- eight terabytes of RAM.


\appendix

% \section{Full Permutability}
% 
% \begin{definition}
%   A regular reduce problem is {\em fully permutable} if, for all
%   permutations $p_1,\ldots,p_n$ of the numbers $1,\ldots,n$,
%   \[
%   \Op{1}_{x_1 \in X'_1}\cdots\Op{n}_{x_n \in X'_n}f(x_1,\ldots,x_n) = \Op{p_1}_{x_{p_1} \in X'_{p_1}}\cdots\Op{p_n}_{x_{p_n} \in X'_{p_n}}f(x_1,\ldots,x_n).
%   \]
% \end{definition}
% 
% \begin{theorem}
%   Full permutability is logically equivalent to exchangeability (and
%   thus block decomposability).
% \end{theorem}
% 
% \begin{proof}
%   ($\Rightarrow$) Given selected split $i$ and partitions $X^{\!L}_i \cup
%   X^{\!R}_i = X_i$, we have
%   \[
%   \begin{array}{rcl}
%     \lefteqn{\GNP(X'_1,\ldots,X'_i,\ldots,X'_n)} \\
%     & = & \displaystyle \Op{1}_{x_1 \in X'_1}\cdots\Op{i}_{x_i \in X'_i}\cdots\Op{n}_{x_n \in X'_n}f(x_1,\ldots,x_n) \\
%     & = & \displaystyle \Op{i}_{x_i \in X'_i}\Op{1}_{x_1 \in X'_i}\cdots\Op{i-1}_{x_{i-1} \in X'_{i-1}}\Op{i+1}_{x_{i+1} \in X'_{i+1}}\cdots\Op{n}_{x_n \in X'_n}f(x_1,\ldots,x_n) \\
%     & = & \displaystyle \Op{i}_{x_i \in X^{\!L}_i}\Op{1}_{x_1 \in X'_i}\cdots\Op{i-1}_{x_{i-1} \in X'_{i-1}}\Op{i+1}_{x_{i+1} \in X'_{i+1}}\cdots\Op{n}_{x_n \in X'_n}f(x_1,\ldots,x_n) \\
%     & & \displaystyle \mbox{} \op{i} \Op{i}_{x_i \in X^{\!R}_i}\Op{1}_{x_1 \in X'_i}\cdots\Op{i-1}_{x_{i-1} \in X'_{i-1}}\Op{i+1}_{x_{i+1} \in X'_{i+1}}\cdots\Op{n}_{x_n \in X'_n}f(x_1,\ldots,x_n) \\
%     & = & \displaystyle \Op{1}_{x_1 \in X'_i}\cdots\Op{i}_{x_i \in X^{\!L}_i}\cdots\Op{n}_{x_n \in X'_n}f(x_1,\ldots,x_n) \\
%     & & \displaystyle \mbox{} \op{i} \Op{1}_{x_1 \in X'_i}\cdots\Op{i}_{x_i \in X^{\!R}_i}\cdots\Op{n}_{x_n \in X'_n}f(x_1,\ldots,x_n) \\
%     & = & \displaystyle \GNP(X'_1,\ldots,X^{\!L}_i,\ldots,X'_n) \op{i} \GNP(X'_1,\ldots,X^{\!R}_i,\ldots,X'_n).
%   \end{array}
%   \]
%   ($\Leftarrow$) Given permutation $p_1,\ldots,p_n$ of the numbers
%   $1,\ldots,n$, with $x^t_{p_i}$ denoting successive elements of
%   $X'_{p_1}$ and $X^t_{p_i} = \{x^1_{p_i},\ldots,x^t_{p_i}\}$, $1 \leq
%   t \leq |X'_{p_i}|$, and the understanding that permuted inputs are
%   mapped to the appropriate arguments of $\GNP$, we have
%   \[
%   \begin{array}{rcl}
%     \lefteqn{\GNP(\{x_{p_1}\},\ldots,\{x_{p_{i-1}}\},X'_1,\ldots,X^t_{p_i},\ldots,X'_n)} \\
%     & = & \GNP(\{x_{p_1}\},\ldots,\{x_{p_{i-1}}\},X'_1,\ldots,X^{t-1}_{p_i},\ldots,X'_n) \\
%     & & \mbox{} \op{p_i} \GNP(\{x_{p_1}\},\ldots,\{x_{p_{i-1}}\},X'_1,\ldots,\{x^t_{p_i}\},\ldots,X'_n).
%   \end{array}
%   \]
%   Induction over $t$ from $|X_{p_i}|$ down to $2$ yields
%   \[
%   \begin{array}{rcl}
%     \lefteqn{\GNP(\{x_{p_1}\},\ldots,\{x_{p_{i-1}}\},X'_1,\ldots,X'_{p_i},\ldots,X'_n)} \\
%     & = & \displaystyle \Op{p_i}_{x_{p_i} \in X'_{p_i}}\GNP(\{x_{p_1}\},\ldots,\{x_{p_i}\},X'_1,\ldots,X'_{p_{i+1}},\ldots,X'_n).
%   \end{array}
%   \]
%   Further induction over $i$ from $1$ to $n$ yields
%   \[
%   \begin{array}{rcl}
%     \lefteqn{\Op{1}_{x_1 \in X'_1}\cdots\Op{n}_{x_n \in X'_n}f(x_1,\ldots,x_n)} \\
%     & = & \GNP(X'_1,\ldots,X'_n) \\
%     & = & \displaystyle \Op{p_1}_{x_{p_1} \in X'_{p_1}}\cdots\Op{p_n}_{x_{p_n} \in X'_{p_n}}\GNP(\{x_{p_1}\},\ldots,\{x_{p_n}\}) \\
%     & = & \displaystyle \Op{p_1}_{x_{p_1} \in X'_{p_1}}\cdots\Op{p_n}_{x_{p_n} \in X'_{p_n}}f(x_1,\ldots,x_n).
%   \end{array}
%   \]
% \end{proof}

\subsubsection*{Acknowledgments}

Use unnumbered third level headings for the acknowledgments. All
acknowledgments go at the end of the paper.

\subsubsection*{References}

References follow the acknowledgments. Use unnumbered third level heading for
the references. Any choice of citation style is acceptable as long as you are
consistent. It is permissible to reduce the font size to `small' (9-point) 
when listing the references.

\small{
[1] Alexander, J.A. \& Mozer, M.C. (1995) Template-based algorithms
for connectionist rule extraction. In G. Tesauro, D. S. Touretzky
and T.K. Leen (eds.), {\it Advances in Neural Information Processing
Systems 7}, pp. 609-616. Cambridge, MA: MIT Press.

[2] Bower, J.M. \& Beeman, D. (1995) {\it The Book of GENESIS: Exploring
Realistic Neural Models with the GEneral NEural SImulation System.}
New York: TELOS/Springer-Verlag.

[3] Hasselmo, M.E., Schnell, E. \& Barkai, E. (1995) Dynamics of learning
and recall at excitatory recurrent synapses and cholinergic modulation
in rat hippocampal region CA3. {\it Journal of Neuroscience}
{\bf 15}(7):5249-5262.
}

\end{document}
