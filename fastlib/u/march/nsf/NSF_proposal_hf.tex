\documentclass[twoside,leqno, 12pt]{article}
\usepackage{ltexpprt}

%\documentstyle[nips07submit_09,times]{article}

\usepackage{amsmath,amssymb}
\usepackage{graphicx}
\DeclareGraphicsRule{.tif}{png}{.png}{`convert #1 `dirname #1`/`basename #1 .tif`.png}

\newcommand{\spcA}{\hspace*{0in}}
\newcommand{\spcB}{\hspace*{.1in}}
\newcommand{\spcC}{\hspace*{.2in}}
\newcommand{\spcD}{\hspace*{.3in}}
\newcommand{\spcE}{\hspace*{.4in}}
\newcommand{\spcF}{\hspace*{.5in}}
\newcommand{\spcG}{\hspace*{.6in}}


%\title{NSF Research Proposal}
%\author{Bill March}
\date{}                                       

\begin{document}
%\maketitle
\begin{center}
\LARGE{Tractable ERI's in Hartree-Fock} \newline
\end{center}
\vspace{-0.5in}
\textbf{Keywords.}  \textit{computational chemistry}

%In a clear, concise, and original statement, present a complete plan for a research project that you may pursue while on fellowship tenure and how you became interested in the topic. Your statement should demonstrate your understanding of research design and methodology and explain the relationship to your previous research, if any. You MUST provide specific details in this essay that address BOTH the NSF Merit Review Criteria of Intellectual Merit and Broader Impacts in order for your application to be competitive. Please refer to the Program Announcement for further information on the NSF Merit Review Criteria. 

%Format: Include the title, key words, hypothesis, research plan (strategy, methodology, and controls), anticipated results or findings, literature citations, and a statement attesting to the originality of the research proposal. If you have not formulated a research plan, your statement should include a description of a topic that interests you and how you would propose to conduct research on that topic. 

%Research topics discussed in your proposed plan will be used to determine eligibility. Refer to the Field of Study eligibility criterion in the program announcement.

this is just the gist that i think makes a good story.  feel free to
use your own language, etc.

first paragraph - short summary:  i am pursuing more efficient methods for quantum 
simulation, by introducing algorithmic ideas from computer science
as well as mathematical ideas from nonparametric estimation theory. 
(this latter part is optional depending on space - see below). 

studying and designing molecules for drug design and materials design
requires simulation of their behavior.  for computational reasons,
simulations at performed at various levels of fidelity, the quantum
level being the most detailed.  there we account for all the 
electrons, etc.  at higher levels, gross approximations are made,
including things like approximating whole side-chains at the protein
folding level.  note that the inability to simulate tractably while
maintaining accuracy is the bottleneck for the protein folding needed 
for drug discovery.

the hartree-fock method is what's used at the quantum level [cite].
even the most accurate methods like David Sherrill's coupled-cluster
methods [cite] build upon hartree-fock.  it has two main computational 
parts, an outer svd and an inner $N^4$ loop.  (show one equation.) for
this reason only tens of atoms can be simulated at the quantum level
of simulation.  we focus our efforts on computing the inner loop
tractably while maintaining very high accuracy.

the FMM [cite] is used here.  however, the FMM was created for
pairwise interactions.  it makes the naive $N^2$ into O(N).  it has been
used in the hartree-fock context by making other hacky approximations
like truncations.  despite the promise of the introduction of the FMM 
to quantum chemistry, it hasn't opened the way to larger molecules.  the
rigorous error bounds of the FMM do not carry over to the entire computation
since the FMM only deals with part of the overall $N^4$ computation.  there 
is no FMM which can deal with 4-way interactions.

[don't go into all the details about the FMM that you did in the current
writeup - first, the exact statements aren't completely accurate; also 
it is not needed]

in [nbody-nips-paper, riegel-tech-report 07], a kind of generalization
of the FMM was shown, which extends to k-way interactions.  the
approach comes from a computer science viewpoint, and can be thought
of as a kind of higher-order divide-and-conquer.  this line of work 
was concerned with statistical problems, such as n-point correlation
functions (which involve a k-way interaction) [gray comp-phys 94] and
kernel density estimation (which can involve gaussian densities)
[lee-paper 06].  the recursive approach can use adaptive 
space-partitioning trees which are more sensitive than oct-trees to
non-uniformly distributed points, and can automatically guarantee
rigorous error bounds on relative error (with respect to the true
quantity) as opposed to absolute error.

describe how it works.  figure can be made more compact by shortening
the lines between the nodes.  also use wrapfig latex package (ask ryan) to 
make it take up much less space.  perhaps show one equation, like the 
hermite expansion for the guassian. %probably not enough space
 
i am investigating the derivation of a fast algorithm for the inner
loop of hartree-fock using this framework.  this involves....  mention
investigating recent data structures from theoretical computer science
such as cover-trees [cite]...  implementing it in software...  testing
on molecules of chemical interest such as...  replacing the Gauss
software to help the quantum chemistry community since good public
codes don't exist...  describe how sherrill and his group are 
providing chemistry intuition, data, etc. to us.

perhaps also get into the idea of using the theory of nonparametric
estimation to select the gaussian bandwidths.  mention the 1/r
approximation, etc. % Almost certainly not enough space

this work has the exciting potential of allowing bigger molecules...
connect back to materials and drug design... etc.


\textbf{Originality.}  This proposal represents my own work and ideas.



\bibliographystyle{abbrv}
\bibliography{NSF_proposal_hf}


\end{document}  