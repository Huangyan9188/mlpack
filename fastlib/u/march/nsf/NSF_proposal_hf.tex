\documentclass[twoside,leqno, 12pt]{article}
\usepackage{ltexpprt}

%\documentstyle[nips07submit_09,times]{article}

\usepackage{amsmath,amssymb}
\usepackage{graphicx, wrapfig}
\usepackage[left=1in,right=1in,top=1in,bottom=1in,nohead,nofoot]{geometry}
\DeclareGraphicsRule{.tif}{png}{.png}{`convert #1 `dirname #1`/`basename #1 .tif`.png}

\newcommand{\spcA}{\hspace*{0in}}
\newcommand{\spcB}{\hspace*{.1in}}
\newcommand{\spcC}{\hspace*{.2in}}
\newcommand{\spcD}{\hspace*{.3in}}
\newcommand{\spcE}{\hspace*{.4in}}
\newcommand{\spcF}{\hspace*{.5in}}
\newcommand{\spcG}{\hspace*{.6in}}


%\title{NSF Research Proposal}
%\author{Bill March}
\date{}                                       

\begin{document}
%\maketitle
\begin{center}
\Large{Tractable Quantum Chemistry Computations}
\end{center}
\vspace{-0.15in}
\textbf{Keywords.}  \textit{Hartree-Fock, higher-order recursion}


%%%%%%%%%% Instructions %%%%%%%%%%%%%%%%
%In a clear, concise, and original statement, present a complete plan for a research project that you may pursue while on fellowship tenure and how you became interested in the topic. Your statement should demonstrate your understanding of research design and methodology and explain the relationship to your previous research, if any. You MUST provide specific details in this essay that address BOTH the NSF Merit Review Criteria of Intellectual Merit and Broader Impacts in order for your application to be competitive. Please refer to the Program Announcement for further information on the NSF Merit Review Criteria. 

%Format: Include the title, key words, hypothesis, research plan (strategy, methodology, and controls), anticipated results or findings, literature citations, and a statement attesting to the originality of the research proposal. If you have not formulated a research plan, your statement should include a description of a topic that interests you and how you would propose to conduct research on that topic. 

%Research topics discussed in your proposed plan will be used to determine eligibility. Refer to the Field of Study eligibility criterion in the program announcement.


%%%%%%%%%%% Introduction %%%%%%%%%%%%%%%%%%
I am investigating more efficient methods for modeling chemical systems at the quantum level.  I intend to apply sophisticated algorithmic techniques from computer science to a fundamental calculation in quantum-level molecular simulations.
%as well as mathematical ideas from nonparametric estimation theory. 
%(this latter part is optional depending on space - see below). 

Applications such as drug and materials design require detailed simulations of the behavior of molecular systems.  However, due to computational constraints, some accuracy must be sacrificed.  
At the most detailed level, quantum mechanical methods model electrons and nuclei separately.  Larger systems can only be handled approximately, with atoms, molecules, or entire proteins treated as interacting units.  While these approximations greatly increase the size of systems that can be modeled, they are simply too inaccurate for the most important applications such as protein folding and drug discovery.   

The basis for all first-principles quantum mechanical simulations is the Hartree-Fock (HF) method \cite{RevModPhys.23.69}.  All other, more sophisticated methods such as coupled cluster build on the HF solution \cite{cc}.  In quantum mechanics, the \emph{wavefunction} governs the behavior of electrons.  Hartree-Fock estimates the wavefunction as a linear combination of specified, usually Gaussian, \emph{basis functions}.  The HF algorithm determines the correct coefficients in this linear combination.  The algorithm can be decomposed into two parts: the formation of the Fock matrix $F_{a b}$, and the solution to an eigenvalue problem using it.  
\begin{displaymath}
F_{a b} = H^{\textrm{core}}_{a b} + \sum_c \sum_d P_{c d} \int \int \phi_a(r_1) \phi_b(r_1) ||r_1 - r_2||^{-1} \phi_c(r_2) \phi_d(r_2) dr_1 dr_2
\end{displaymath}
Since $a, b, c$ and $d$ range over the basis functions, this computation is na\"{i}vely $O(N^4)$, where $N$ is the number of basis functions, thus making it the rate-limiting step.  By using Gaussian basis functions, the integrals here have closed-form solutions, which can be represented as Gaussians.  I will focus on computing this matrix tractably while maintaining very high accuracy and bounded error.


Since the integrals above are the quantum analog of Columbic interactions, the Fast Multipole Method has been applied to this problem \cite{grngard}.  For pairwise interactions, which are na\"{i}vely $O(N^2)$, the FMM computes them in $O(N)$ time.  In HF, the Continuous FMM \cite{white_continuous} requires truncations and other approximations which lose the rigorous error bounds of the original FMM.  Additionally, the CFMM still does not directly attack the $O(N^4)$ computation, since the FMM cannot deal with four-way interactions.  Therefore, the FMM has not permitted quantum simulations on large systems. 


%% Just cite  Riegel and Gray.  Title. GA Tech Tech report 2007
There is a method, which can be viewed as a generalization of the FMM, that extends to $k$-way interactions \cite{gray_nbody, techrep}.  These approaches arise from a computer science viewpoint and are a form of higher-order divide-and-conquer.  This line of work has tackled statistical problems, such as the $N$-point correlation (which involves a $k$-way interaction) \cite{comp.phys94} and kernel density estimation \cite{NIPS2005_570}.  This recursive method can automatically guarantee rigorous error bounds and can use adaptive space-partitioning trees which are more sensitive to structured data than the oct-trees of the FMM.  

For a $k$-way computation, the algorithm considers interactions between $k$ nodes in the tree at a time.  Using the bounding boxes of the nodes, it determines bounds on the possible values of the computation between points.  In previous applications, multipole expansions similar to those in the FMM have been used to improve these bounds.  When these bounds indicate the computation will be within a specified relative error tolerance, further computation is \emph{pruned} and the computation between each $k$-tuple of points in the $k$-nodes is approximated by the computation between the nodes.  Otherwise, the algorithm recursively considers interactions between the children of the nodes.  
%perhaps show one equation, like the hermite expansion for the guassian. - probably not enough space
 \begin{wrapfigure}{r}{0.45\textwidth}
\includegraphics[width=0.45\textwidth]{3nodes_multipole.eps}
\caption{Three nodes $Q_I, Q_J, Q_K$ with bounding boxes.  The arrows represent the approximated interactions.}
\label{three_nodes}
\end{wrapfigure}

\vspace{-0.2in}
I am investigating a fast algorithm for computing the Fock matrix using this algorithmic framework.  This involves deriving the bounds on computations mentioned above and determining the best tree to organize the basis functions.  Once the algorithm is designed, I will implement it in software and test against existing methods.  With the support of David Sherrill and his research group, I test my algorithm on systems of real chemical interest.  I will also integrate my algorithm with his open-source quantum chemistry code \cite{psi3}, thus providing a useful alternative to the current commercialized programs.  


Given the past success of algorithms employing this framework on similar problems, I believe this work will provide a tremendous speedup over existing methods.  This will allow simulations on much larger systems than were previously feasible at this level of detail.  Additionally, detailed simulations are often used to determine appropriate approximations for simpler ones, so my algorithm will help expand the accuracy of these models as well.  By allowing new simulations, and improving existing ones, it should be possible for researchers to make great progress in understanding protein interactions, leading to better drug design.  

\textbf{Originality.}  This proposal represents my own work and ideas.



\bibliographystyle{abbrv}
\bibliography{NSF_proposal_hf}


\end{document}  