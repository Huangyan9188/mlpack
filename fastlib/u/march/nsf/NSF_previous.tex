\documentclass[twoside,leqno, 12pt]{article}
\usepackage{ltexpprt}

%\documentstyle[nips07submit_09,times]{article}

\usepackage{amsmath,amssymb}
\usepackage{graphicx}
\DeclareGraphicsRule{.tif}{png}{.png}{`convert #1 `dirname #1`/`basename #1 .tif`.png}

\newcommand{\spcA}{\hspace*{0in}}
\newcommand{\spcB}{\hspace*{.1in}}
\newcommand{\spcC}{\hspace*{.2in}}
\newcommand{\spcD}{\hspace*{.3in}}
\newcommand{\spcE}{\hspace*{.4in}}
\newcommand{\spcF}{\hspace*{.5in}}
\newcommand{\spcG}{\hspace*{.6in}}


%\title{NSF Research Proposal}
%\author{Bill March}
\date{}                                       

\begin{document}
%\maketitle
\begin{center}
\LARGE{NSF Graduate Research Fellowship Previous Research} \newline
\Large{Bill March}
\end{center}


%Describe any scientific research activities in which you have participated, such as experience in undergraduate research programs, or research experience gained through summer or part-time employment or in work-study programs, or other research activities, either academic or job-related. Explain the purpose of the research and your specific role in the research, including the extent to which you worked independently and/or as part of a team, and what you learned from your research. In your statement, distinguish between undergraduate and graduate research experience. 

%If you have no direct research experience, describe any activities that you believe have prepared you to undertake research. At the end of your statement, list any publications and/or presentations made at national and/or regional professional meetings. 

Opening paragraph: quick summary of things I've worked on/looked into
Tie this into interests established before in the first essay, set up interest in proposal
Graph coloring
Computational Geometry
Fast algorithms
Computational biology and chemistry

%1/2 page on other things I've looked at
%	Worked on graph coloring
%	Other computational geometry - convex hulls
%	Robot stuff
%	Protein folding
%	Computational chemistry
%1.5 pages on hierarchical clustering
%	Explain kd-trees while I'm at it, can refer to this later
	

%%%%%%%% Hierarchical Clustering %%%%%%%%%%
\textbf{Hierarchical Clustering and EMST}

\textbf{Context.}
Euclidean Minimum Spanning Trees are fundamental structures in computational geometry.  They have been applied to many things.  Additionally, the EMST is equivalent to a hierarchical clustering of the underlying points.  This form of clustering is common in biology and cosmology.  

The problem has been around for a long time.  There are algorithms, but none are great.  Now, there are gene microarrays and the SDSS, which scientists might want to apply EMST/hierarchical clustering to.  So, we need better scalable algorithms to make this happen.

I approached Alex in Spring 2006 about research opportunities.  He suggested it might be possible to apply some of his techniques to the EMST problem.  I began working on this problem with him over the summer.  Did it myself.  Worked with his grad students on learning the other algorithms.  Figured out the algorithm, wrote the code, did the experiments on my own.


\textbf{Details.}
Computing nearest neighbors - should be able to copy a lot of this from the paper

Include a picture of a kd-tree, etc.  All the details really exist to set up the proposal.


Results
Include some of the better charts

What did I learn?  How does it help the proposal?

%%%%% On-line graph coloring %%%%%%%%
My role: read and thought, learned some stuffs
Learned I wanted something more directly applicable



%%%%%% List publications and presentations %%%%%%%%
Submitted EMST paper.


\bibliographystyle{abbrv}
\bibliography{NSF_proposal}

\end{document}