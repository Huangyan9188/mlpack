\documentclass[twoside,leqno, 12pt]{article}
\usepackage{ltexpprt}

%\documentstyle[nips07submit_09,times]{article}

\usepackage{amsmath,amssymb}
\usepackage{graphicx}
\DeclareGraphicsRule{.tif}{png}{.png}{`convert #1 `dirname #1`/`basename #1 .tif`.png}

\newcommand{\spcA}{\hspace*{0in}}
\newcommand{\spcB}{\hspace*{.1in}}
\newcommand{\spcC}{\hspace*{.2in}}
\newcommand{\spcD}{\hspace*{.3in}}
\newcommand{\spcE}{\hspace*{.4in}}
\newcommand{\spcF}{\hspace*{.5in}}
\newcommand{\spcG}{\hspace*{.6in}}


%\title{NSF Research Proposal}
%\author{Bill March}
\date{}                                       

\begin{document}
%\maketitle
\begin{center}
\Large{Personal Essay} 
\end{center}

%NSF Fellows are expected to become knowledge experts and leaders who can contribute significantly to research, education, and innovations in science and engineering. The purpose of this essay is to demonstrate your potential to satisfy this requirement. Your ideas and examples do not have to be confined necessarily to the discipline that you have chosen to pursue. 

%Describe any personal, professional, or educational experiences or situations that have prepared you or contributed to your desire to pursue advanced study in science, technology, engineering, or mathematics. Describe your competencies and evidence of leadership potential. Discuss your career aspirations and how the NSF fellowship will enable you to achieve your goals. 


%%%%%%%%%%%%% Outline %%%%%%%%%%

%Intro
%	Interests 
%		Why chemistry and biology?
%			That's where all the action is
%			They're the least well understood and most relevant 
%		Studied mathematics as the best way to really understand science at many levels
%		See computers as the real way to get at sciences - something like what Alex suggested is fairly key, I think
%		Some kind of anecdote might be helpful here, but not strictly required

%Experiences
%	FASTlab
%		Have talked and presented my research to the lab
%		Reading group presentations
%		Been a part of larger research team - Need to be specific here - need some more ideas
%			Discussions with Ryan
%			Working with Dongryeol and Arkadas
%		Presented to other science labs - this is important
%		Interpreting between the chemistry people
%	Teaching
%		Think I did an okay job with this one above, can probably just revise it
%	Fraternity president - add this if there's space
%		Lots of organization - meetings, events, other people
%		Communication between inside and outside groups
%		Keys - teamwork, leadership, communication
%		
%Goals/Conclusion
%	Being a professor
%		
%	How does this tie back to the beginning?
%
% 	How does NSF help with this?
%		

%%%%%%%%%%%%%%% Intro %%%%%%%%%%%%%%%%%%%
%%% Opening #1 %%%%%
%I see graduate school as the biggest opportunity of my life.  I have always read widely.  I always want to learn more, often about any topic.  I read widely in science and mathematics, but also history, literature, current events, anything I can get my hands on.  My home is filled with books, but there is one I always have prominently displayed: "The Bad Seed" by William March.  I don't particularly care for the book, and my name being the same as the author's is just a coincidence.  I always like seeing my own name among all the books.

%Now that I am in grad school, that finally seems within reach.  I have a few classes and other obligations, but most of my time is now free to discover things.  Now I have the chance to write the book.  

%
%% Not great, try to stay more focused on chemistry and biology from the beginning
%Although my interests are very broad, I have always been drawn to science.  I chose to study mathematics instead, which lead into computers.  Throughout my studies, I was always searching for applications, which has lead me to computational chemistry.  



%%%%%%%% Opening #2 %%%%%%%%%%%
% Maybe include something about Harvey's class here?
% Make the opening/motivation for chemistry and biology something about the open field thing Alex suggested

Opening sentence?

I believe that chemistry and biology offer more opportunities and will more profoundly change the world in the near future than any other sciences.  However, simply studying these areas is not sufficient to handle the changes.  Real progress requires drastically new methods and approaches, both of which are accomplished by computers.  However, most existing computational approaches are either primarily designed by computer people, which makes the range of their applications limited in the world of science, or they are designed by scientists, which means they fail to take advantage of the full power of applied mathematics and algorithm design.

I intend to continue studying sophisticated algorithms and data structures in order to apply them to problems of pressing interest in computational chemistry.  In order to accomplish this, I am pursuing a Ph.D. in computer science at Georgia Tech.  I am working in the Computational Science and Engineering Division and being advised by Alexander Gray and David Sherrill.  I am able to work closely with faculty in both algorithm design and scientific computing.  My eventual goal is to continue my  research as a university professor. 

%%%%%%%%%%%%% Teaching %%%%%%%%%%%%%%%%%%%
%%% TA experience
% Pretty good, should be able to use this in basically this form
% Do a bit more to drive home the points that: 1) I can communicate my findings to a broad audience, 2) interpret and communicate research findings
% Include that I got good reviews consistently, was asked to take on more courses

%TA - teaching, education, communication - need to be clear
%	Lots of experience with explaining technical things to non-technical audience
%	Teaching linear algebra to uninterested freshmen who don't see why it's important
%	Relates to trying to communicate algorithms to scientists, science to algorithm people
%	Often found it easiest to interest students using applications they cared about
%		Graphics, electrical engineers, whatever
\section{Teaching} 

For five semesters as an undergraduate, I worked as a teaching assistant in the School of Mathematics for courses on linear algebra, calculus, and differential equations.  I was responsible for grading tests and answering students questions.  However, the part of the job I really enjoyed was teaching the recitation sections.  Twice a week, I was responsible for leading a section of about 40 students for an hour.  In general, the professors I worked for gave me few instructions, so I was free to organized the class as I saw fit.  

I loved the challenge of communicating complex ideas to (often disinterested) students.  I quickly came to appreciate the difficulty of this task.  In order for any of them to stick, every concept had to be carefully motivated.  Complex details had to be carefully summarized.  Too many details, no matter how interesting I found them, quickly lost all of the students.  Too few, and they were unable to handle subtle problems later.  I found students were often best motivated by examples in applications.  A student interested in computer graphics would doze through linear algebra until I motivated reflection of light as a matrix computation.  Realizing the power of this method motivated me to learn more about these applications.  I began bombarding friends in various fields for examples of eigenvectors and Laplace transforms in their classes.  As I became more involved in my own research, I often used it as a source of examples.  
%% Does this need any more?  Not now, but maybe later. . .

%Through teaching, I found that I did not understand some of the concepts as well as I had thought, which motivated me to learn even more.  
%On a few occasions, I couldn't resist the temptation to digress into some interesting feature of the topic at hand.  


%%%%%%%%%%%%%%%%% FASTlab %%%%%%%%%%%%%%%%%%%%
%%% Experiences with FASTlab %%%
%Other things I have done with FASTlab:
%	Helping other people with papers
%	Giving talks on my research
%	Discussing applications of my algorithm
%	Discussing varied research ideas
%	Presenting ideas to other labs: Skolnick, Sherrill
%	Interpreting in the discussions with chemistry people

\section{FASTlab}

In my last summer as an undergraduate, I started working with Alex Gray's research group, the FASTlab.  Initially, I was researching a faster algorithm for finding Euclidean Minimum Spanning Trees.  Although I had done research before, it was an individual project.  Now, I found myself interacting with several graduate students with backgrounds in mathematics, sciences, and computer science, all working on fast algorithms for various applications.  

% Maybe have a section on individual work, maybe that's covered in the next section

% teamwork - needs revision
% do more name dropping?
\textbf{Teamwork.}  I quickly learned to appreciate working with the other students.  I was trying to apply some of the lab's previous work to a new problem.  Before I could even begin, I had to fully understand the published algorithms.  I often sought out the other students with questions.  When working on my own algorithm, I gained many valuable suggestions from the other students that helped me overcome difficulties.  Even simple tasks like learning the lab's code base were much easier in a group environment.  

Since my initial work with the FASTlab, my own research has expanded to include more scientific applications.  Although I generally pursue my own projects, I rarely miss opportunities to discuss them with the other members of the lab.  While investigating opportunities in computational chemistry, I realized I needed a better understanding of quantum mechanics.  I did extensive reading on my own, but I often sought the help of the lab's physicist postdoc.  
%maybe one more example here, this just falls a little flat

%maybe something about becoming an expert in my own areas within the lab

% presenting/communicating
\textbf{Communication.}  With the FASTlab, I have gained valuable experience in presenting my ideas to 
diverse audiences.  Each week, a student gives a one hour talk on his or her research.  I have presented on my work on minimum spanning trees and on some investigations in computational biology.  Presenting the technical details involved in simulating protein folding was particularly challenging, given that the audience was mostly from a mathematics and computer science background.  

Additionally, we are engaged in collaborations with several other research groups.  We often give presentations on our work to other research groups, both in computer science and other sciences.  We have presented to Prof. Jeffrey Skolnick's computational biology group and Prof. David Sherrill's computational chemistry group.  On both occasions, I was responsible for preparing slides and presenting about my work and ideas, which involved communicating material from statistics and algorithm design to scientific audiences.  Both of these meetings led to ongoing collaborations with these researchers.


%% Stress that I'm translating in meetings with David and Alex


%% Maybe try to add something about exploring my own ideas
% Definitely need to say something about the freedom to follow my own ideas and interests in chemistry.
% Want to say that the environment is supportive, currently pursuing some projects with people, etc.


%%%%%%%%%%%% Goals and interests %%%%%%%%%%%%%%%%%%%%%%%%%%

% Add something about NSF's role in helping me do this

\section{Goals}

After finishing my degree at Georgia Tech, I plan to continue my research as a university professor.  While there are many ways to pursue research, I believe that academia offers the greatest freedom to impart my ideas to students and colleagues, interact with scientists, and ensure that my work has the greatest possible benefit.  As a professor, I will also have the opportunity to continue teaching.  In addition to the challenge this provides, I think it is an excellent opportunity to further solidify new ideas by presenting them convincingly to more than just experts.  

%Needs more revision, basic idea is there
With the NSF fellowship, I will enjoy more freedom to seek problems and collaborators in the sciences.  Since I would not have to rely on a professor for funding, I would have more time to read deeply in the sciences and ensure that my algorithms are truly useful and applicable.  



%	
%Fraternity president - probably going to need to include this after all
%	leadership, setting goals, teamwork, organization, communication
%	

% IMPORTANT:
% Make sure and include something about Harvey's class!!!

%%%%%%%%%%%%% Cover this stuff %%%%%%%%%%%%%%%%%%%%%%
%
%Intellectual Merit:

%Plan and conduct research - have done it before, will do it again
%	Came up with chemistry connections
%Work as a member of a team - working with the FASTlab, fraternity president
%Work independently - did my own algorithm
%Interpret and communicate research findings - talks, discussions with chemistry people, presentations to biology group

%
%Broader Impacts:

%Benefit society - computers are the way to find drugs, fight disease, study evolution, etc.
	% include the thing Alex suggested - really clever computer science hasn't been applied here because of the high barrier for entry
%Enhance scientific understanding - I'm the one who can apply deep computer science and applied math here, because I'm willing and capable of learning the science
%Integrate research and education - have already given talks to my research group on computational biology, teaching others about QM, HF, AT, etc.
%	teaching - have used examples from my research to motivate students?
%Encourage Diversity - ???  I adopted an African child.  Everyone is made of atoms.  


\bibliographystyle{abbrv}
\bibliography{NSF_proposal}

\end{document}