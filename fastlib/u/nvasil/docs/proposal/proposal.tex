
%% Note: If you are using this template for a proposal, pass the proposal option to
%% the class.  Then the toplevel section will be \section, not \chapter.  In either case
%% the \appendix command, which should only be used in the EndMatter environment, can be considered
%% to be at the same toplevel (either section or chapter}.

%% NOTE: Hyperref package doesn't yet completely work with the proposal option.  It compiles
%% but the pdf bookmarks aren't correct.

%% Please report bugs/problems to djallred@gmail.com

%% Begin Preamble Info
\documentclass[12pt,letterpaper,doublespaced,ETD,dvips,proposal]{gtthesis}

\title{Nearest Neighbor methods in machine learning with applications to speech}
\author{Nikolaos Vasiloglou II}
\copyrightyear{2007}
\graddate{August 2007} %Appears at bottom of title page
\approvaldate{July 2007} %Appears on signature page (not needed for a proposal)

% Add people to signature list (not needed for a proposal)
\addadvisor{David Anderson}{Assoc. Professor, School of ECE}{Georgia
Institute of Technology}

\addadvisor{Alexander Gray}{Assist. Professor, School of CS}{Georgia
Institute of Technology}

\addchair{Dr. Sitting Bull}{Professor, School of ECE}{Georgia
Institute of Technology}

\addreader{Dr. Wayne Geronimo}{Asst. Professor, School of
ECE}{Georgia Institute of Technology}

\addreader{Dr. David Goliath}{Asst. Professor, School of
ECE}{Georgia Institute of Technology} \addreader{Dr. Marky
Mark}{Asst. Professor, School of ECE}{Georgia Institute of
Technology}

\addreader{Dr. Larry A. Stooge}{Asst. Professor, School of
ECE}{Georgia Institute of Technology}
\bibfiles{template}

%%Packages to use
\usepackage{graphicx}
\usepackage{insfig}
\usepackage{hyperref}

%%%% These commands set the default depth the TOC should be expanded
%%   out to for the chapters and appendices
%%
\setchaptertocdepth{2}
\setappendixtocdepth{2}

%%%% To rename different sections of the document use these
%%   set???string{ ...string...} commands where ??? is replaced
%%   by one of the following three letter codes
%%
%%   toc = table of contents   chp = chapter
%%   lof = list of figures     app = appendix
%%   lot = list of tables      prt = part (i.e. volume)
%%   bib = bibliography        abs = abstract
%%   ind = index               los = list of symbols
%%   ded = dedication          ack = acknowledgement
%%   glo = glossary,list of terms
%%
%%   Regardless of capitalization all will become capitalized in
%%   the final document.
%%
\settocstring{Table of Contents}
\setlofstring{List of Figures}
\setlotstring{List of Tables}
\setbibstring{References}
\setindstring{Index}
\setdedstring{Dedication}
\setglostring{List of Terms}
\setchpstring{Chapter}
\setappstring{Appendix}
\setprtstring{Volume}
\setabsstring{Summary}
\setlosstring{List of Symbols}
\setackstring{Acknowledgment}

%%%% Set the default page style for each main enviroment
%%
\setfrontpagestyle{plain}
\setbodypagestyle{plain}
\setendpagestyle{plain}

\begin{document}
\pagestyle{plain}

\begin{FrontMatter}
\contents %Generates the TOC, LOF, and LOT, glossary and LOS
%% Below are some optional environments that can be used here
\end{FrontMatter}

\begin{Body}
\section{Introduction}
\label{intro}
The start of stuff. \par

\section{Origin and history of the problem}
\label{origin}
%\inssssfig{campanile.eps}{campanile.eps}{campanile.eps}{Test
%side-by-side-by-side figure}{3cm}{3cm}{3cm}
\section{Preliminary Research}
\label{prelim}
\subsection{The Nearest Neighbor Problem}
One of the classical problems in statistics and in machine learning
as is the nearest neighbor problem. Given a set of points in $S =
{x_i, x_i\in\Re^d, i=0\dots N}$ and a distance metric $M(x_i, y_i),
x_i, y_i \in S$  the nearest neighbor of $x_i \in \Re^d$ is
$argmin(M(x, y_i)), y_i \in S$. In many cases the problem of finding
all nearest neighbors appears. This problem can be formalized as
find $argmin(M(x, y_i)), y_i \in S, \forall x \in S'$. The problem
of finding nearest neighbor is can be solved in a "naive" way, by
computing all the distances between the query point and the
reference points. The complexity is $O(N)$. Assuming that the number
of the query points $N$ is the same as the number of reference
points then the complexity of the naive method is $O(N^2)$. An
efficient way to reduce the complexity is to use multidimensional
trees. In the next sections we present kd-trees and ball-trees that
tend to be the most successful.

As a clarification we mention that this is the problem of the exact
nearest neighbor. There are algorithms addressing the problem of
approximate neighbor. Local Sensitivity Hashing is one of the
algorithms that deal with this problem.

\subsubsection{Multidimensional Trees}
The purpose of of multidimensional trees is to create a hierarchical
partition of a dataset so that data points are grouped close to
their neighbors, which are usually the points of interest, and far
away enough from other points. It has been proven that there is not
a globally optimal partition strategy and it is heavily dependant on
the distribution of the data points as well as on the dimension. In
this section we present a particular class of multidimensional
trees, the binary trees and more specifically kd-trees and ball or
metric trees. The key concept in binary trees is partitioning a set
recursively in two sets at every step.

\subsubsection{Kd trees}
The first multidimensional tree in the literature was Kd-tree. After
30 years it still has comparable performance and in some case even
better with modern trees. At every level a partition (pivot)
dimension is chosen according to a user defined criterion and the
data is split in two subsets according to this dimension. Kd trees
are off-line trees. In order to build a Kd-tree all the data must be
available before building the tree in contrast to other trees that
can dynamically add or delete points.

Several strategies have been recommended for finding the splitting
dimension. The most popular are splitting on the dimension with the
maximum variance or maximum range. Both of them are heuristics and
there is no underlying theory behind it that guarantees better
performance of the trees. Another issue once the splitting dimension
is found, is the actual numeric value of split value. In other words
the ratio of the points going in the two resulting sets. The mid
value of the range or the mid value of the variance will in general
give two sets that in most of the cases will have different number
of points. Another strategy is to find the value that gives equally
sized sets. This strategy in general gives balanced trees, while the
first one leads to unbalanced trees. In the following sections we
show that empirically it is more preferable to have unbalance splits
on the tree, since it favors pruning in the process of finding the
nearest neighbors.

\emph{Data structures for Kd trees}

 Kd-tree consists of two types of data structures, the nodes and the
 leafs. The basic data needed for the nodes are the
 hyper-rectangles. Hyper-rectangles are the k-dimensional rectangles
 that describe the minimum bounding box  for a given set of points.
 Leafs hold additionally the data points. The number of points on a
 leaf is user defined. As we will see in the following sections the
 tree method for finding the nearest neighbor is faster than naive
 only after a threshold of points, (experimentally it has been found
 to be 30). This threshold is used as the maximum number of points
 per leaf. Bounding boxes (Hyper-rectangles) and points are the minimum amount
 of data required to be stored on nodes on the leaf. Extra
 information about the data such as the centroid of the box, higher
 order moments, etc also called Cached Statistics can speed up
 tremendously algorithms other than nearest neighbor.


\subsubsection{Ball trees}
One of the main disadvantage of kd-trees is that they can only
handle problems associated with the euclidian metric, or metrics
that can be casted in euclidian metric such as the hamming distance.
In many cases although the data are expressed in cartesian
coordinates the notion of similarity is beyond the euclidian sense.
For example the cosine distance is used in many applications of
machine learning such as document retrieval.

Metric or Ball trees give solution to this problem. Instead of using
hyper-rectangles, we use hyper-spheres. Hyper-spheres are
generalized k-dimensional spheres:
\begin{equation}
x,x_0 \in \Re^k , d(x,x0)=r, r\in \Re
\end{equation}
where $d(x, y)$ defines a metric in a k dimensional space.

At every split two points that are far away are chosen as anchor
points. Every other point of the initial set is classified according
to the minimum distance from the anchor points. The choice of anchor
points is done heuristically. One method that works in practice is
choosing a point randomly and then find the furthest point. This is
the first anchor point. The next anchor point is chosen as the
furthest point from the first.
\subsubsection{Nearest Neighbor Algorithm}
The purpose of using a tree in the nearest neighbor problem is to
gradually find bounds for the neighbor distance and prune parts of
the search space. In general trees give bounds for data partitions
that can be used in other problems that we will see in future
sections, such as fast gaussian summation.

Initially we start the search from the root node. At each step we
follow recurse down to the child node that contains the query point.
In kd-trees we check the pivot dimension of the query point and if
it is smaller than the pivot value we recurse on the left child
otherwise on the right. In ball trees we recurse to the child that
is closer to the query point. The recursion continues until we reach
a leaf. Then the distances of the query point and the points on the
leaf are computed. The closest point is a candidate nearest
neighbor. Moreover we know that the actual nearest neighbor can not
be further than the current nearest neighbor. So in the end of the
top down recursion we have an upper bound for the nearest neighbor.
Note that the top down recursion is logarithmic to the number of
points assuming a balanced tree. The nest step is to backtrack. At
each step in backtracking we check if the candidate nearest neighbor
is in the current leaf/node. There are two ways to check that. First
of all when we are on a leaf we check weather the hyper-sphere
centered around the query point with radius equal to the distance
 between the query point and the candidate nearest neighbor
 $d_{min}$ intersects the bounding box. If not then this means it is
 the nearest neighbor. If it does then we have to check all the
 bounding boxes that intersect the hyper-sphere. For kd-trees and
 ball trees it is fairly easy to show detect that.

 In general for natural data-sets where data tend to cluster it is
 possible to prune very well  and achieve logarithmic performance in
 search. In some cases when the partitioning is not very good, trees
 behave poorly and eventually have to search almost the whole
 data-set and they behave worse than the naive method since they
 have extra recursion overhead.
\subsubsection{All nearest Neighbors}
The problem of computing all nearest neighbors appears often in many
machine learning problems, usually in the training phase. As we will
see in the following sections computation of the kernel matrix
involves all nearest neighbor computations. The all-nearest neighbor
problem is a special case of a more general class of problems called
N-body problems.

Having in mind the tree based algorithm it is natural to assume that
finding all nearest neighbors can be solved by executing N nearest
neighbor queries resulting in an $O(N\log )$ complexity. We will
call this as the single tree all nearest neighbor algorithm. However
it is not difficult to notice that there are redundant computations.
For example consider we have a query set stored on a kd tree and a
reference set stored on another kd-tree as well. If we do the top
down recursion for a point then we notice that most likely this top
down recursion might be exactly the same for all the points in the
same query leaf. It is also possible that the top down recursion
might be the same for a whole query subtree.

This observation leads us to the conclusion that instead of
comparing distances between points in the search it is more useful
to compare bounding boxes. The algorithm is  a four-way recursion.
In general the dual tree algorithm as we call it since the query and
the reference set lie on trees (they might share the same tree if
the query and the reference set are the same), gives linear
complexity over the number of data. This is empirical complexity and
\subsubsection{Fast kernel summation with trees}
Kernel summation is on of the most common tasks in kernel methods.
The general formula
\begin{equation}
\label{kernel_sum}
  G(x)=\sum_{q=0}^{N} w_i\exp(\frac{(x-x_q)^2}{\sigma^2})
\end{equation}

appears in many cases, with the most interesting case the
eigenvalue/eigenvector computation. The kernel summation is a very
computationally intensive. For Gaussian kernels there are
approximations that can speed up computations orders of magnitude.
We will briefly discuss them here. Besides computation saving we can
save memory as well. The gaussian kernel has infinite support so the
kernel matrix $K, K(x,y)=\exp(\frac{(x-y)^2}{\sigma^2}$ is not
sparse. Numerical iterative methods for eigenvector computation are
based on the general iteration $x_n=x_{n-1}+\lambda Kx_{n-1}$, where
the multiplication of each row of $K$ with $x_{n-1}$ is a kernel
summation problem. So we don't really need to store the kernel
matrix, we just have to compute the kernel sum for every row using
$x_{n-1}$ as a the wight vector.

The gaussian can be approximated with the help of Hermite
polynomials. The Hermite polynomials $H_n(t)$ are defined by the
Rodrigues formula:
\begin{equation}
  H_n(t)=(-1)^n\exp(t^2)D^n\exp(-t^2), t\in \Re
\end{equation}
where $D=d/dt$.

The gaussian can now be expressed as
\begin{equation}
\exp^(-(t-s)^2)=\sum_{n=0}^{\inf}\frac{s^n}{n!}h_n(t)
\end{equation}

where $h_n(t)$ is defined by
\begin{equation}
h_n(t)=\exp(-t^2)H_n(t).
\end{equation}


\subsection{Kernel methods}
The kernel or Gram matrix is an informational representation for a
set of data. There are a different interpretations for the kernel
matrix. The most mathematical interpretation is the dot product
matrix. Every element of the kernel matrix is the dot product
between the two data points. It is a positive semidefinite matrix
and it can be considered as the adjacency matrix of a graph where
the points are nodes. This graph can also be considered as an
approximation of the surface that data lie on. Positive
semidefinitness is the only property required for the gram matrix.
In most of the methods that we discuss in this section, we prefer
the interpretation of the dot product matrix, in the general Hilbert
sense. The dot products in most of the cases are non-linear.
\subsubsection{Kernel principal component analysis}
Kernel Principal Component Analysis (KPCA), is an extension of
principal component analysis where the data are mapped into a
different feature space through the kernel. In Principal Component
Analysis the goal is to find the principal components for  the
covariance matrix $C$ for a given dataset $S={x_j, x_j\in \Re^d, j=1
\dots M}$, where $C=\frac{1}{M}\sum_{j=1}^{M}x_j x_j^T$. If we
define a mapping
\begin{equation}
\Phi : x \rightarrow X , x \in \Re^d, X\in \Re^f
\end{equation}

usually $f>>d$. So usually the nonlinear mapping, maps the data in a
higher dimensional space sometimes in an infinite dimensional space.
At this point it should be clarified that the mapping increases the
extrinsic dimension of the data not the intrinsic. In kernel PCA we
want to find the principal components of the covariance matrix
\begin{equation}
\tilde{C} = \frac{1}{M}\sum_{j=1}^{M}\phi(x_j)\phi(x_j)^T
\end{equation}

It turns out that the mapping $\Phi$ doesn't have to be known
explicitly, all we need is the kernel that represents the dot
product.
\begin{equation}
k(x, y) = \phi(x)\phi(y)^T
\end{equation}
For some kernels there is an analytic factorization while for others
there is not. Not every function can be a valid kernel (dot
product). There are conditions that must be fulfilled. Basically for
every dataset they  must give a valid kernel matrix as described in
previous sections.

Kernel PCA has many cousins such as Laplacian eigenmaps, Diffusion
maps Isomap, Local Linear embedding

\subsubsection{Customized Kernels}
As we described in the previous section there are infinite functions
that can act as valid kernels. The most popular kernels are:
\begin{enumerate}
  \item The gaussian kernel $k(x, y)=e^{-\frac{||x-y||^2}{\sigma^2)}}$
  \item The polynomial kernel $k(x, y)=(xy)^n$
  \item The epanechnikov kernel $k(x, y)=1-||x-y||^2/\sigma^2, ||x-y||<\sigma, else 0$
  \item The k-nearest neighbor kernel
\end{enumerate}

Although these are valid kernels for any kernel method including
Kernel PCA they are not appropriate for every task. In some cases we
want to map data in higher dimensional spaces preserving distances
(isometry) while in other cases we want to make small distances
smaller and send high distances to infinity (clustering). The
gaussian kernel for example tends to behave better in clustering
since it fades out very quickly. On the other hand it fails to
unfold the data manifold so that the intrinsic dimension can be
estimated. Manifold unfolding is a very important procedure since it
can reveal the true dimension of the dataset.

In [?] the authors introduced the idea of building the kernel matrix
from scratch without using any fixed kernel. As we have already
mentioned the kernel matrix has to be a positive semidefinite
matrix. Initially we have to define a k neighborhood  for every
point. So at first we have to connect points to their neighbors and
build a initialize the non zero values of the matrix to 1. Then we
maximize the trace of the kernel matrix under the constraint that
the the distances in the kernel matrix are preserved. This leads to
a semidefinite programming problem.
\subsubsection{Stochastic optimization}
Stochastic optimization or on line learning. On line svd is one key
aspect.
\subsubsection{Support Vector Machines}
\section{Proposed Research}
\label{proposed}

The goal of this thesis is to make kernel methods scalable.
Efficient computation of the kernel matrix for large amounts of
data. Special emphasis is given in high dimensional data $d>30$. One
of the goals is to use Kernel PCA for unfolding the speech manifold.
In other words estimate the intrinsic dimensionality of speech by
using big databases like timit, wall street journal and broadcast
news. Implicitly dimensionality reduction will lead to new feature
generation we will explain it in the following sections. The
ultimate goal is to do speech recognition by nearest neighbor
search.
\subsection{Large scale trees, for out of core memory}
Kd-trees have been proven to be be quite successful in
multidimensional indexing and turn out outperform several other
trees in statistical processing of data mainly in all nearest
neighbor problem, n-point correlation etc. In this paper we address
different implementations of kd-trees for data that don't fit in the
RAM. We show whether virtual memory behaves better   than paging of
nodes on the disk.

Kd trees have been used quite successfully in statistics for
speeding up statistical algorithms (nearest neighbor, 2 point
correlation  over large dataset. Many other trees like ball-trees,
vantage trees, cover trees etc have been developed as a rivalry but
they haven't really shown any significant improvement, their
performance is most of the times comparable to the kd-trees. After
30 years they still remain the state of the art for developing
statistical algorithms. In most of the cases kd trees have been used
for datasets that fit in the main memory.

During the same period the database community developed structures
for multidimensional indexing mainly used for database applications.
Some examples of successful tree structures are R-trees,
$R^{*}$-trees, X-trees and TV-trees. The main incentive for these
trees was to handle large volumes of data, dynamic insertions,
deletions and random accesses.

The size of the datasets for statistical processing is increasing
faster than the availability of  main memory so it is necessary to
investigate the performance of kd trees and various optimizations
over cached implementations. Unfortunately R-tree like structures
cannot be used or even compared to kd-trees since they are optimized
for different problems. It turns out that batch loading of spatial
trees can improve their performance. There are splitting strategies
in some of them (X-trees) that can guarantee non overlapping
hyper-rectangles which is considered to be a good thing for the
nearest neighbor performance. Eventually kd-trees can be considered
as a special case of spatial trees, so there is no point in
comparing to them.

In this paper we consider some caching implementation strategies for
kd-trees. The first approach is the use of virtual memory through
memory mapped files. The basic advantage is the the programming
simplicity, since data can be accessed through normal pointers. The
operating system takes care of the page faults. The second approach
is to take care of the cache so that we can use bigger page sizes
than the default 4096 in Linux. Page fault is handled on the
application level. In order to minimize the io cost between disk and
main memory we used the TPIE library that does efficient block
transfers.

Another parameter that affects the performance of the trees is the
layout of the tree on the memory, that has to be in such a way that
it guarantees good locality in terms of the access pattern of some
basic kd tree algorithms, such as all  nearest neighbors, n-point
correlation. There are different ways to build a kd-tree: a)Depth
first b)Breadth first and c)The combination of them building k-depth
trees breadth first. It has been shown experimentally that
unbalanced kd-trees tend to give better pruning performance, we also
consider balanced tree splits since they can give better locality.

The last implementational issue is prefetching.  Queries in
databases are random most of the time so prefetching doesn't really
make sense. On the other hand statistical algorithms on trees
generate queries that are highly correlated, so we can take
advantage of that. Taking advantage of the multicore architecture it
is possible to do non blocking system call for prefetching.

In this paper we do not address the problem of building cache
efficiently trees since this issue has been extensively addressed in
the literature, with the most successful algorithm the buffer tree.

\subsection{Memory layout of Kd-trees} There are many ways to build a
kd tree, depth first, breadth first and the combination for both. It
turns out that breadth first does not give good locality for
dual-tree algorithms. We will show experimentally that the hybrid
method gives the best results in terms of page faults and depth
first has comparable performance. Another parameter of kd-trees that
has effect in the cache performance is the splitting algorithm.
Mid-point splits  guarantee balanced trees, but do not always give
good performance in terms of punning.

\subsection{System architecture} There are several ways to implement
a disk kd-tree. As it has been stated above, speed is an essential
feature that we are particularly interested as well ass portability.
Since building the tree for large scale data might take a lot of
time it is very important to be able to reload the tree and perform
other operations. Although it is desirable to keep the tree as
generic as possible so that we can use it for different applications
this might sacrifice the performance. In our implementation we
avoided virtual functions, structures that require memory
reallocations. In order to keep it generic we used simple concepts
of C++ template programming. Another issue that is of great
importance is memory locality. There are five basic memory
structures involved in a kd tree, nodes, leafs, data points node
cached statistics and hyper-rectangles. Some of them can be combined
but it is always good to preserve the abstraction hierarchy Tree,
Node, bounding Box, Node Statistics, and Points. There is also
another one which can be treated separately, the result. This can be
the nearest neighbor(s), the kernel density estimation etc. They can
be stored either on the tree or on external structures linked from
the tree nodes. There are two major ways to lay out the tree on the
memory. The first method is to use a separate cache for each data
structure and the other one is to use  a single cache for
everything. In some cases the output of the algorithm might as well
be orders of magnitude larger than the actual tree. For example
k/range nearest neighbors can give pretty big result sets for large
k or range.

\subsection{Memory Manager Architecture} In order to test different
memory manager strategies, we used smart pointers that would
collaborate with their corresponding memory allocators. Memory
Allocators implement a set of basic functions such as malloc. One of
the fundamental issues is to make sure the allocated memory for each
structure had the correct alignment. It is widely known that wrong
memory alignment can decrease the performance or even cause a
program crash. an easy and portable way to compute the memory
alignment is to use the simple macro depicted in ??. Another issue
that has to be addressed is page fault handling. In the following
section we describe two different memory managers we implemented and
in section ? we compare their performance.

\subsection{Memory Mapped Files} In this approach we used a single
memory mapped file big enough to fit the tree. The length of the
file has to be known ahead, but it is not hard to estimate it. The
memory is allocated linearly and since there are no reallocations or
deletions there is no need to worry about memory fragmentation.
Paging is taken care by the operating system. The page size is fixed
4096B for Linux. Customizing the page size is something desirable
but this shouldn't be of great worry since very soon Linux will
accommodate for variable page size [?]. This implementation is
pretty simple and memory access is achieved with normal pointers. It
is also efficient since all the paging is done in lower levels of
the operating system

\subsection{User defined cache, with TPIE} In this implementation a
block of memory is preallocated and page faults are handled in an
LRU sense, while the block transfers between main memory and the
disk are done with the TPIE engine. TPIE is a portable library for
efficient block transfers. It 's highly abstracted to accommodate
for different transfer models. It is currently implementing memory
mapped mode, Linux read/write and C fread/fwrite. In the future it
will implement the parallel disk model as well. The advantage of
this memory manager is that we can control/define several features,
such as page size, page replacement algorithm and disk/memory
transfers. The main disadvantage is the address translation that has
to be done on the application level and slows down the system. In
order to minimize this effect we sacrificed a little bit of the
programming interface simplicity. So the address translation is
explicitly done in the initialization of the pointer. So the pointer
includes a universal address (integer) and a pointer to the main
memory. Upon request the internal pointer p points to the data in
the main memory. As it is obvious this particular memory segment
might be paged out during the execution. To avoid that the
programmer explicitly locks the particular page as long as the
pointer is in use and then it is explicitly unlocked when the data
pointing to is not necessary anymore. This is  policy is always
prone to deadlocks if the cache size is small and the page size is
large. Another possible case of deadlock is when the tree building
algorithm does not guarantee good locality in the access pattern of
the dual-tree algorithm. In practice we never came across a
deadlock, since depth first and kdepth first give good enough
locality for N-body algorithms. But even in the case of bad locally
we could make local copies of the data and not lock the pages,
sacrificing the performance of the algorithm. It turns out though
that the locking/unlocking strategy is quite efficient because it
avoids copies and it minimizes the cost of address translation. This
coding scheme is sensitive to bugs because the programmer might
accidently keep pages locked, which can have an immediate effect in
the program performance in the same way memory leaks behave. For
this reason we trace locks/unlocks in debug mode to ensure that the
program is free of this type of bugs. Another important issue is
marking the modified pages so that we save unnecessary copies from
main memory to disk. We used the SIGSEGV library along with mprotect
to address efficiently this. Pages are write protected in memory and
if a write occurs a signal is raised and the page is marked as
modified.

\subsection{Customizing Kernels with fast
gradient descent}

As we described in section ? it is possible to customize the kernel
for a given dataset, through semidefinite programming. Although this
method is very promising and gives very good results it is not
scalable. Semidefinite programming is solved with the intrinsic
point method which is of polynomial complexity.

\subsection{Scalable Support Vector Machines}
\subsection{Computing Gaussian Kernel Matrix with trees}
\subsection{Intrinsic dimension of speech}


\subsection{Speech recognition with nearest neighbor search}

In this paper we present a computationally efficient means of
characterizing different features for speech processing tasks. The
presented method can be used to compare the performance of features
free of interference from back-end recognition engines that might be
skewed to favor a particular feature. Understanding the intrinsic
characteristics of features would enable us to make the right choice
of features for a given task.
\subsection{Introduction} Many features have been proposed for speech
recognition, with Mel Frequency Cepstrum Coefficients (MFCC)
\cite{quatieri2002dts} being the most popular. Other features have
also been developed claiming better performance than MFCC in limited
tests over different data sets. The most common metric is the speech
recognition word accuracy over TIMIT or any other data set.
Unfortunately most of the speech recognition systems are
specifically tuned for MFCCs and it is difficult to have objective
comparison since systems always tend to favor them. In this paper we
will introduce statistical algorithms that can scale easily for
large data sets. One very interesting statistical measure for
features that can be the basis for many machine learning algorithms
is all nearest neighbors, which is a special case for the general
N-body problem. The brute force method is of $O(N^2)$  complexity
and virtually impossible to scale for large data sets. One way to
speed up computations is to order the features on a multidimensional
tree, such as kd or ball tree \cite{bentley1975bst},
\cite{moore2000ahu} which provide on the average $O(N\log_2N)$
complexity for all nearest neighbor search. A faster algorithm using
2 trees has made it possible to compute all nearest neighbors in
linear time \cite{gray2000nbp}.  In this paper we demonstrate the
performance of the fast all nearest neighbor algorithm over the
whole TIMIT database using two different features, MFCC and Noise
Robust Auditory Features (NRAF) \cite{ravindran:inr}. Since the data
are phonetically labeled, the results are used for evaluating the
k-nearest neighbor classifier performance. This is also a good
measure for the compactness and separability of the phonetic classes
for these two features.

Another fundamental problem in speech generation theory is the
estimation of the intrinsic dimension of the speech manifold. Recent
theories have proven that although speech is described with high
dimensional representations, it is actually embedded on a low
dimensional subspace \cite{jansen2006ifa}. The most easy but
inaccurate method to estimate the embedding of speech in a low
dimensional space is Principal Component Analysis (PCA), which
suffers from many problems, specially because of the presence of
non-linearities on the data. On the contrary many other methods for
dimensionality reduction have been proposed. These are kernel PCA
\cite{scholkopf:nca}, diffusion maps, Laplacian eigenmaps
\cite{coifman2006dm} etc. All these methods operate on a local level
and require the computation of the nearest (range or k) neighbors of
every point. The brute force and single tree methods have made the
cost prohibitive for large data sets, but the dual tree method makes
possible to evaluate the proximity graph between the points and
apply any of the above mentioned methods.

\subsection{Speech features} In our implementation, MFCC features
were extracted using the freely available code from \emph{Voicebox}.
The speech signal was divided into frames of length 25 msec, with 50
\% overlap and 13 MFCC coefficients (including the zeroth
coefficient) where extracted from each frame. Twenty-seven filters
were used to group the frequency bins into critical bands.

The NRAF features were derived from a model of the early auditory
system~\cite{shamma_main}. The input signal was passed through a
bandpass filter bank. We then performed a difference operation
between adjacent frequency channels. This is followed by a half-wave
rectification and a smoothing filter. We compute a discrete cosine
transform (DCT) of the logarithm of the output of the smoothing
filter to obtain the NRAFs. The filter bank consisted of 27
1/4$^{th}$ octave filters tuned from 76 Hz to 6887 Hz. The smoothing
(temporal integration) was done over 25 msec and sampled at a rate
of 80 Hz (in order to match MFCCs). As in the case of MFCCs, the
first 13 coefficients were used for further processing.

In both cases the delta and acceleration coefficients were extracted
and appended to each feature vector, increasing the dimensionality
from 13 to 39.

\subsection{Fast N-body methods for feature comparison} One of the
fundamental problems in statistics is the nearest neighbor
computation for every point in a dataset. The brute force also known
as the naive method requires $O(N^2)$ computations. Ordering the
multidimensional data on a binary tree, gives $O(\log (N)$ search
time per point and $O(N\log N)$ in total. There is a plethora of
binary multidimensional trees ball tree, kd tree, cover tree etc. In
our experiments we used the kd tree. A detailed description of kd
trees can be found at \cite{moore-tutorial}. The main concept is to
recursively partition data in disjoint rectangles. Each tree node
contains a k-dimensional rectangle (hyper rectangle) that bounds all
the points that are contained in the subtree. Only leafs contain
points. In Fig.~\ref{kdtree} an example of a kd tree in 2 dimensions
is given, while the algorithm for computing the nearest neighbor for
a test point is depicted in table Fig.~\ref{dualtree_algorithm}. The
main idea is to recursively find the leaf that contains the query
point then compute a candidate nearest neighbor on the leaf. As the
algorithm back tracks it determines if it is necessary to search
other subtrees for a nearest neighbor.

The use of dual trees in the all nearest neighbor search has proven
to give significant speed-up in all-nearest neighbor computations.
The basic idea is that the whole query set is structured on a tree.
In our case the query tree and the reference tree are the same. The
core of the algorithm relies on the fact that if the current maximum
nearest neighbor distance in the query node is smaller than the
distance between the query and the reference node distance then the
reference node can be pruned. This saves a huge amount of
computation in most of the cases since whole  subtrees can be pruned
out for a large subset of points at once. The all nearest neighbor
search is described algorithmically in Table
\ref{dualtree_algorithm}. The complexity of the algorithm, on
average, turns out to be linear.

\begin{table}
\begin{verbatim}
recurse(q : KdTree, r : KdTree) {
  if (can_prune(q, r)) {
    summarize(q, r) /* prune */
  } else if (IsLeaf(q) and IsLeaf(r)) {
    do_all_pairs(q, r)
    /* at leaves we must resort to n^2 */
  } else if (<should recurse on q>) {
    recurse(q.left, r)
    recurse(q.right, r)
  } else {
    recurse(q, r.left)
    recurse(q, r.right)
  }
}
\end{verbatim}
\caption{Pseudo-code for the dual-tree all nearest neighbor
algorithm} \label{dualtree_algorithm}
\end{table}



\insfig[kdtree]{kdtree.eps}{A two dimensional kd-tree}{8cm}


\insfig[dualkdtree]{dualtree_recursion.eps}{Simulation of the dual
tree algorithm}{8cm}


\subsection{Kernel PCA on a manifold} Let  $(\Gamma,\mu)$ be a
measure space, where $\Gamma$ is a finite set of $N$-dimensional
points and $\mu$ is a counting measure that represents the
distribution of the points on the data set. In other words $\Gamma$
is a submanifold of $\Re^{n}$. Assume that the geometry of $\Gamma$
is defined by a kernel $k(x,y)$. The kernel $k(x,y)$ measures the
degree of similarity between two points $x,y$. The kernel satisfies
the following conditions:
\begin{itemize}
    \item   $k$ is symmetric: $k(x,y)=k(y,x)$,
    \item   $k$ is positivity-preserving: for all $x,y$ in
$\Gamma, k(x,y)\geq 0$
    \item   $k$ is positive semi-definite: for all bounded
functions $f$ defined on  $\Gamma$,
\[
\int_{\Gamma} \int_{\Gamma} k(x,y)f(x)f(y)d\mu(x)d\mu(y)\geq 0\]
\end{itemize}
In this paper it is assumed that $\Gamma$ is a subset of the
Euclidean space $\Re^{n}$. So for $x,y\ni\Gamma$ the kernel
(similarity measure) is a function of the Euclidean distance
$\parallel x-y\parallel:$
\[
 k(x,y)=\eta(\frac{\parallel x-y\parallel}{h})
\]
The eigenvalue analysis of the graph with the adjacency matrix
$k(x,y)$ is the basis for embedding the data to a lower dimensional
space.
\[
    k(x,y)=\sum_{j=0}^{M} \lambda_{j}\phi_{j}(x)\phi_{j}(y)
\]

where $M$ is the number of data points, $x$ and $y$ are nodes of the
graph and $\phi_j$ is the $jth$ eigenvector that corresponds to the
$\lambda_j$ eigenvalue.

One problem is that exhaustive computation of the kernel will lead
to a full matrix, which makes the computation of the eigenvector
$O(N^2)$. Fortunately very few $(x,y)$ pairs have a non zero kernel
product since it depends on their proximity. So again the
computational bottleneck for kernel PCA is the computation of
nearest neighbors which can be solved by the dual tree algorithm in
linear time.

\subsection{Experimental Results} In order to illustrate the
applicability of the algorithms described above, the TIMIT and Noisy
TIMIT (NTIMIT) database were chosen. Our goal is to test our
algorithms in terms of scalability and time. Phoneme classification
with the nearest neighbor classifiers was chosen as the evaluation
task. The task was repeated for both set of features, namely, MFCC
and NRAF. We also illustrate the spectrum of kernel PCA for the two
different features.
\subsection{All nearest neighbor performance}
The timing results are illustrated in table \ref{timing}. The all
20-nearest neighbor method with the dual tree algorithm on kd-trees
is 30 to 50 times faster than naive in terms of time. It took about
3.6 hours for MFCCs and 2.2 hours for NRAF for all 20-nearest
neighbors. It is well known that the pruning performance of the
trees depends on the intrinsic dimension of the dataset. A simple
way to get an estimation of the intrinsic dimension is the
eigenvalue distribution of the Principal Component Analysis. As seen
from Fig.~\ref{pca} NRAFs have a more compact distribution than
MFCCs.
\subsection{Comparison of NRAF and MFCC}
Two types of experiments were done for the comparison of the
features. The first one is the Leave One feature Cross Validation.
So for every feature we find the k-nearest neighbors and check
whether the majority has the same label as it \footnote{Instead of
using the 61 different phonetic classes of TIMIT we used the 39
categories of the CMU dictionary}. In tables \ref{kneighborTIMIT}
and \ref{kneighborNTIMIT}, we demonstrate the performance of MFCC
and NRAF for TIMIT and NTIMIT for various k. NRAF perform better
than MFCC, specially in the noisy case.

In the second experiment we do Leave-One-Speaker-Out cross
validation. For every feature we find the k-nearest neighbors that
don't belong to the same speaker and check if the majority has the
same label as it. The results are illustrated again in
\ref{kneighborLOOCVTIMIT} and \ref{kneighborLOOCVNTIMIT}. The
results drop significantly and it interesting that MFCCs perform
better than NRAFs. The reason why the performance drops
significantly is because we haven't done any covariance
normalization on the data. It is well known in the speech
recognition literature that mean and variance normalization improves
recognition as it removes the speaker bias. It turns out that
variance normalization broadens significantly the PCA spectrum which
is tightly connected with the performance of kd-trees. We tried
variance normalization but the all nearest neighbor performance was
very slow. Another option is the Mahalanobis distance on a metric
tree, but again the covariance normalization would have an immediate
effect on the performance.

It turns out that variance normalization is a way of decorrelating
the MFCC/NRAF dimensions, which breaks the manifold structure. This
is a suggestion that we should move on with higher dimensional
transformations that remove the speaker bias intrinsically.

In fig.~\ref{kernel_pca} the kernel PCA spectrum is illustrated. The
Gaussian kernel with adaptive local bandwidth was used for the
20-nearest neighbors. The bandwidth is proportional to the nearest
neighbor distance and every point is connected with its 20 nearest
neighbors. It turns out that the kernel PCA spectrum is very wide.
This is a result of the fact that 20 nearest neighbors are not
enough to model correctly the manifold. It is an indication that
more neighbors should be used. We experimented with the bandwidths
and it turns out that no further improvement can be achieved by
tuning it. High bandwidths tend to give more compact spectrum, so
even in the extreme case of infinite bandwidth no more significant
improvement is shown for 10 nearest neighbors.

\subsection{Discussion}
It has been shown previously that NRAFs tend to be more robust to
noise compared to MFCCs~\cite{ravindran:inr}. From
Tables~\ref{kneighborTIMIT} and~\ref{kneighborNTIMIT} it is clear
that NRAFs do provide substantial improvement over MFCCs specially
in noisy environments. However, it is very interesting to note that
MFCCs outperform NRAFs in all Leave-One-Speaker-Out cross validation
tests. This would suggest that MFCCs are better than NRAFs at
removing inter-speaker variability. Inter-speaker variability
usually is the result of incorporating pitch information (usually
encoded in the higher coefficients of the decorrelation transform)
and it is very much possible that in ensuring NRAFs and MFCCs have
the same number of channels we may have introduced pitch information
into the NRAF features. Perhaps using lesser number of NRAF
coefficients or different filter spacing, the inter-speaker
variability could be removed.


\instbfig[pca]{timit_pca.eps}{ntimit_pca.eps}{Principal Component
Analysis of TIMIT (top) and NTIMIT (bottom) for MFCC (solid line)
and for NRAF (dashed line).}{8cm}{8cm}


\insfig[kernel_pca]{kernel_pca.eps}{Kernel PCA spectrum for TIMIT
MFCC features}{8cm}

\begin{table}[h!]
\footnotesize{ \centering
\begin{tabular}{|c|c|c|c|}
  \hline
  % after \\: \hline or \cline{col1-col2} \cline{col3-col4} ...
  feature & Naive & Single Tree & Dual Tree \\
  \hline
  MFCC & 100 & 4.3 & 3.6 \\
  NRAF & 100 & 2.9 & 2.2 \\
  \hline
\end{tabular}
 \caption {CPU time (in hours) for evaluating all 20 nearest neighbors on
  TIMIT database}
  }
  \label{timing}
\end{table}

\begin{table}[htb]
\footnotesize{ \centering
\begin{tabular}{|c|c|c|c|c|c|}
  \hline
  % after \\: \hline or \cline{col1-col2} \cline{col3-col4} ...
  feature & 1-nearest & 3-nearest & 4-nearest & 7-nearest & 10-nearest \\
  \hline
  MFCC & 62.2\% & 70.4\% & 69.7\% & 65.3\% & 63.4\% \\
  NRAF & 66.9\% & 74.1\% & 73.2\% & 67.3\% & 64.7\% \\
  \hline
\end{tabular}
\caption{Nearest k-neighbor classfier for TIMIT database} }
\label{kneighborTIMIT}
\end{table}

\begin{table}[htb]
\footnotesize{ \centering
\begin{tabular}{|c|c|c|c|c|c|}
  \hline
  % after \\: \hline or \cline{col1-col2} \cline{col3-col4} ...
  feature & 1-nearest & 3-nearest & 4-nearest & 7-nearest & 10-nearest \\
  \hline
  MFCC & 41.4\% & 55.5\% & 54.5\% & 50.7\% & 49.4\% \\
  NRAF & 55.0\% & 65.1\% & 64.0\% & 57.5\% & 54.5\% \\
  \hline
\end{tabular}
\caption{Nearest k-neighbor classfier for NTIMIT database} }
\label{kneighborNTIMIT}
\end{table}

\begin{table}[htb]
\footnotesize{ \centering
\begin{tabular}{|c|c|c|c|c|c|}
  \hline
  % after \\: \hline or \cline{col1-col2} \cline{col3-col4} ...
  feature & 1-nearest & 3-nearest & 4-nearest & 7-nearest & 10-nearest \\
  \hline
  MFCC & 44.0\% & 56.3\% & 56.3\% & 54.7\% & 54.6\% \\
  NRAF & 42.2\% & 53.5\% & 53.7\% & 52.4\% & 52.4\% \\
  \hline
\end{tabular}
\caption{Nearest k-neighbor classfier for
  Leave One Speaker out Cross Validation in TIMIT database}
} \label{kneighborLOOCVTIMIT}
\end{table}

\begin{table}[htb]
\footnotesize{ \centering
\begin{tabular}{|c|c|c|c|c|c|}
  \hline
  % after \\: \hline or \cline{col1-col2} \cline{col3-col4} ...
  feature & 1-nearest & 3-nearest & 4-nearest & 7-nearest & 10-nearest \\
  \hline
  MFCC & 32.0\% & 46.7\% & 46.5\% & ?44.3\% & 43.8\% \\
  NRAF & 30.1\% & 42.9\% & 42.9\% & 41.2\% & 40.1\% \\
  \hline
\end{tabular}
  \caption{Nearest k-neighbor classfier for Leave One Speaker Out
  Cross validation in  NTIMIT database}
} \label{kneighborLOOCVNTIMIT}
\end{table}

\subsection{Acoustic Environment adaptation}

\section{Work remaining to be done}
\label{remains}

\section{Facilities and Equipment Needed}
\label{facitlies}
\end{Body}

\begin{EndMatter}
\appendix{This is my first appendix}
\appendix{This is my second appendix}
%% Below are some optional environments that can be used here
\bibliographystyle{ieeetr}
\references % Generates the bibliography page
\index % Generates the index - should be last
\end{EndMatter}

\end{document}
