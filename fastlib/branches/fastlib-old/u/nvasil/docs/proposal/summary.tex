\documentclass[12pt]{article}

\begin{document}
\pagestyle{plain}
\begin{center}
    \textbf{Proposal Summary}
\end{center}

\textbf{Goal:} Compute the intrinsic dimension of several speech
representations. Fast computation for kernel matrix. Fast
optimization of the kernel matrix. Nearest neighbor speech
recognition.

\textbf{Contributions:}
\begin{enumerate}
  \item Fast computation for kernel matrix.
  \item Fast optimization of the kernel matrix.
  \item On-line kernel adaptive filters.
\end{enumerate}

\textbf{Specific tasks (Done):}
\begin{enumerate}
  \item \textit{The Nearest Neighbor Problem} Approached the problem
  with trees that give the exact nearest neighbor. We could have also
  considered the approximate nearest neighbor problem that can be
  solved with other methods. But still trees is the best approach
  and also it is not yet known whether approximate nearest neighbor
  gives better performance.
  \item \textit{Multidimensional Trees} Studied developed and
  optimized kd-trees and ball trees. We didn't consider other types
  since comparative study has shown that they are still the state of
  the art.
  \item \textit{All nearest Neighbors} Implemented the best
  algorithm (dual tree) algorithm.
  \item \textit{Large scale trees, for out of core memory} Followed
  different caching practices for trees that don't fit in the main
  memory. Optimized code. Showed that the dual tree algorithm has
  very good locality structure.
  \item \textit{Fast k-nearest neighbor kernel computation} It is
  based on direct application of the all-nearest neighbor algorithm
  \item \textit{tuning the kernel bandwidth} Developed and algorithm
  for tuning non-parametrically the gaussian kernel in spectral
  clustering algorithms such as kernel-pca and diffusion mapping
  \item \textit{Phoneme classification with nearest neighbor
  classifier} Experimented with MFCC and NRAF features on the whole
  TIMIT database and managed to get up to 60\% correct phoneme
  classification only by looking at the nearest neighbor. The method
  can scale up linearly.
\end{enumerate}

\textbf{Specific tasks (To do):}
\begin{enumerate}
  \item \textit{Customizing Kernels with semidefinite programming}
  There is an existing method for optimizing the kernel matrix based
  on semidefinite programming. There are certain optimizations that
  can be done along with the dual tree algorithm for linear scaling,
  such as application of stochastic optimization
  \item \textit{Fast kernel summation with trees, Kernel Matrix
  inversion} Direct computation of the kernel(Gaussian) kernel leads
  to a non-sparse matrix. There is a way though to compute the
  inverse/eigenvalues without explicitly computing the kernel
  matrix. Several optimizations can be done for high dimensions
  \item \textit{Speech recognition with nearest neighbor search}
  Implement a merge reduce scheme to create different descriptions
  of speech. Use nearest neighbor algorithms to replace gaussian
  mixtures in hidden markov models.
  \item \textit{Parallelize the code to handle larger datasets} The
  goal is to measure the intrinsic dimension of the broadcast news
  dataset. Test it with fast gaussian summation and with optimized
  kernels.
\end{enumerate}

\end{document}
